%% Generated by Sphinx.
\def\sphinxdocclass{report}
\documentclass[a4,10pt,french]{sphinxmanual}
\ifdefined\pdfpxdimen
   \let\sphinxpxdimen\pdfpxdimen\else\newdimen\sphinxpxdimen
\fi \sphinxpxdimen=.75bp\relax
\ifdefined\pdfimageresolution
    \pdfimageresolution= \numexpr \dimexpr1in\relax/\sphinxpxdimen\relax
\fi
%% let collapsible pdf bookmarks panel have high depth per default
\PassOptionsToPackage{bookmarksdepth=5}{hyperref}

\PassOptionsToPackage{warn}{textcomp}
\usepackage[utf8]{inputenc}
\ifdefined\DeclareUnicodeCharacter
% support both utf8 and utf8x syntaxes
  \ifdefined\DeclareUnicodeCharacterAsOptional
    \def\sphinxDUC#1{\DeclareUnicodeCharacter{"#1}}
  \else
    \let\sphinxDUC\DeclareUnicodeCharacter
  \fi
  \sphinxDUC{00A0}{\nobreakspace}
  \sphinxDUC{2500}{\sphinxunichar{2500}}
  \sphinxDUC{2502}{\sphinxunichar{2502}}
  \sphinxDUC{2514}{\sphinxunichar{2514}}
  \sphinxDUC{251C}{\sphinxunichar{251C}}
  \sphinxDUC{2572}{\textbackslash}
\fi
\usepackage{cmap}
\usepackage[T1]{fontenc}
\usepackage{amsmath,amssymb,amstext}
\usepackage[francais]{babel}


\usepackage{times}


\usepackage[Sonny]{fncychap}
\ChNameVar{\Large\normalfont\sffamily}
\ChTitleVar{\Large\normalfont\sffamily}
\usepackage{sphinx}

\fvset{fontsize=auto}
\usepackage{geometry}


% Include hyperref last.
\usepackage{hyperref}
% Fix anchor placement for figures with captions.
\usepackage{hypcap}% it must be loaded after hyperref.
% Set up styles of URL: it should be placed after hyperref.
\urlstyle{same}

\addto\captionsfrench{\renewcommand{\contentsname}{Table des matières}}

\usepackage{sphinxmessages}
\setcounter{tocdepth}{1}


%\usepackage[titles]{tocloft}
%\cftsetpnumwidth {1.25cm}\cftsetrmarg{1.5cm}
%\setlength{\cftchapnumwidth}{0.75cm}
%\setlength{\cftsecindent}{\cftchapnumwidth}
%\setlength{\cftsecnumwidth}{1.25cm}
\newcommand{\seminarytitle}{Développement Web}
\newcommand{\customizeinfos}{}


\title{Création du site internet destiné au Comité Etudiant Humanitaire du Collège du Sud}
\date{janv. 15, 2023}
\release{Version intermédiaire}
\author{Capucine Böhning}
\newcommand{\sphinxlogo}{\vbox{}}
\renewcommand{\releasename}{Collège du sud, Travail de maturité}
\makeindex
\begin{document}

\ifdefined\shorthandoff
  \ifnum\catcode`\=\string=\active\shorthandoff{=}\fi
  \ifnum\catcode`\"=\active\shorthandoff{"}\fi
\fi

\pagestyle{empty}
\sphinxmaketitle
\pagestyle{plain}
\sphinxtableofcontents
\pagestyle{normal}
\phantomsection\label{\detokenize{index::doc}}



\chapter{Introduction}
\label{\detokenize{introduction:introduction}}\label{\detokenize{introduction::doc}}
\sphinxAtStartPar
En tant qu’étudiante en 3ème du Gymnase, j’ai eu l’opportunité de participer à mon premier projet de programmation de site internet pour l’association Candide, et ce dans le cadre de mon TM. Ce travail m’a permis de mettre en pratique mes connaissances et de développer de nouvelles compétences en matière de développement web et tout ce que cela implique.


\section{Motivation}
\label{\detokenize{introduction:motivation}}
\sphinxAtStartPar
Je trouve ce projet motivant par la diversité des taches demandées et des connaissances requises pour mener à bien le développement d’un site internet. Le fait qu’il s’agisse de fabriquer quelque chose de A à Z le rend d’autant plus intéressant. Tout au long de ce travail, je pense évidemment acquérir de nombreuses notions de design et informatiques, mais également apprendre à gérer un projet en entier, ce qui ne semble pas être tâche facile ! De plus, ce site sera utile pour le comité lorsqu’il sera question se présenter au près d’autres associations pour un éventuelle collaboration. Étant membre de Candide, je suis encore plus motivée à produire un travail de qualité


\section{Intérêt du Projet}
\label{\detokenize{introduction:interet-du-projet}}
\sphinxAtStartPar
Comme préalablement stipulé, je fais partie de Candide. C’est pourquoi j’ai pu me rendre compte de certains besoins que le comité peut rencontrer lors de contact avec des associations. Notamment d’un seul endroit regroupant toutes les informations à savoir à propos de son activité, de ses actions passées, de ses objectifs et surtout d’un moyen de contact simple.
Aujourd’hui, le comité Candide a comme seul moyen de communication son compte Instagram, malheureusement ce n’est pas suffisant pour regrouper toutes les informations. De plus, cette plateforme n’est pas professionnellement reconnue, ce qui dessert à l’association. Donc Candide ne dispose pas vraiment d’outils similaires et c’est pourquoi un site web simple d’utilisation et permettant de retrouver simplement les informations nécessaires serait très utile.


\section{Les difficultés à venir}
\label{\detokenize{introduction:les-difficultes-a-venir}}
\sphinxAtStartPar
Les principales difficultés résideront dans la création en elle\sphinxhyphen{}même du site internet, ainsi que dans l’apprentissage des différents langages de programmation utilisés. Pour la création du design il me faudra également découvrir de nouveaux outils, ceci semble aussi être un savoir qui prendra un certain temps à assimiler. Dans le but de mieux surmonter ces quelques difficultés et les autres qui ne sont pas citées et qui sont inattendues, il sera essentiel de bien se renseigner avant de se lancer dans les différentes étapes de réalisation du projet.\\
Plusieurs technologies différentes vont être nécessaires au bon développement de ce site. Premièrement, pour le design du site, l’utilisation de Adobe XD sera particulièrement nécessaire pour créer des maquettes qui permettront au Comité Candide de se projeter et d’apporter les modifications demandées le plus simplement possible. Ensuite, le site internet sera lui\sphinxhyphen{}même programmé en HTML et CSS, ce qui permettra d’avoir un code le plus organisé possible. Enfin, pour le CMS servant à la modification facile du contenu du site, des outils tels que contentful seront employés. Je suis convaincue que ce projet sera un réel succès et permettra à l’association Candide de disposer d’un outil performant pour communiquer efficacement avec les autres associations.


\chapter{Technologies Utilisées}
\label{\detokenize{technologies-utilisees:technologies-utilisees}}\label{\detokenize{technologies-utilisees::doc}}

\section{Les Maquettes}
\label{\detokenize{technologies-utilisees:les-maquettes}}
\sphinxAtStartPar
Tout d’abord pour l’élaboration des maquettes, j’ai dû utiliser un logiciel qui m’était préalablement inconnu, Adobe XD. En effet, cet outil permet de créer des prototypes de site web avec previews. Il a joué un rôle important dès le début du projet, notamment lors de discussion avec le Comité. À plusieurs reprises, nous avons pu discuter des besoins de leurs besoins mais également de leurs envies et de leurs gouts. Il était aussi très important que le design du site corresponde au mieux à ce que l’association voulait, c’est pourquoi beaucoup de modifications ont été apportées et qu’il était nécessaire de commencé par la création de maquette par le biais de Adobe XD.


\section{La programmation}
\label{\detokenize{technologies-utilisees:la-programmation}}
\sphinxAtStartPar
En ce qui concerne les framework, j’ai utilisé du HTML et CSS. Ils semblaient être les langages de programmation les plus adéquat, mais aussi les plus courants lorsqu’il s’agit de coder un site web statique.\\
Le HTML est un langage de balisage, qui sert donc à définir les différents contenus des pages. Ce langage est privilégié pour indiquer de quoi est constituée chaque page et ce qu’elles doivent afficher. Le HTML va permettre d’indiquer qu’un contenu est un texte qui n’est qu’un paragraphe, un texte qui est un titre de niveau 1 dans notre page, une liste, un lien, une images, ….
Le CSS, quant à lui, sert à définir le style du contenu. Il est un langage très pratique car il regroupe toutes les informations des contenus du HTML. Il permet également d’apporter des modification de manière simple et rapide, car les changements seront directement mise\sphinxhyphen{}à\sphinxhyphen{}jour sur le HTML relié. Grâce au CSS, il est aussi possible d’ajouter des effets plus spécifique, comme des effets de transparence, des animations lorsque le curseur de la souris passe sur un bouton ou simplement des couleurs et format de textes, d’images, etc….


\chapter{Prototypage}
\label{\detokenize{chapitre-02:prototypage}}\label{\detokenize{chapitre-02::doc}}

\section{Recherches Artistiques}
\label{\detokenize{chapitre-02:recherches-artistiques}}
\sphinxAtStartPar
Tout d’abord, j’ai dû entrer en communication avec Candide afin de comprendre leurs attentes et leur audience cible. J’ai pu ainsi savoir quelles informations devaient être mises en avant sur le site et l’image que Candide aimerait donner de son association. J’ai alors créé un moodboard à partir de ce qui avait été dit et en mettant en avant les valeurs qui semblent être importantes pour le comité.\\
Dans ce design plusieurs notions sont mises en avant. D’abord l’aspect protection de la nature, par le vert qui domine et des images florales ou abstraites qui viennet ajouter un peu de douceur. Tout cela rappelle l’importance de la protection de l’environnement pour Candide.
Ensuite, la police utilisée est dans le style des années 70, style qui se veut vivant, respirant et changeant avec l’environnement. Ces valeurs emblent être effectivement liées à celles de l’association.\\
De plus, les couleurs de ce site internet seront très importantes pour mettre en avant la bien\sphinxhyphen{}être et la bonne humeur que Candide veut apporter au Collège du Sud, ainsi que la bienveillance et l’inclusivité de son comité. Aussi l’idée de cadrillage pourrait apporter un certain rythme et dynamiser le site.

\begin{figure}[htbp]
\centering
\capstart

\noindent\sphinxincludegraphics[width=1.000\linewidth]{{MoodBoard}.png}
\caption{Mood Board}\label{\detokenize{chapitre-02:id1}}\end{figure}


\section{La structure du Site}
\label{\detokenize{chapitre-02:la-structure-du-site}}
\sphinxAtStartPar
Le choix des différentes pages du site a aussi demandé de longues discussions.

\sphinxAtStartPar
Tout d’abord la question s’est portée sur les informations qui doivent demeurer sur la page d’accueil du site.\\
Premièrement une courte description de ce qu’est Candide semblait nécessaire pour que chacun puisse savoir facilement de quoi il s’agit. Ensuite, comme Candide est un association qui base ses valeurs sur l’action, il est important de parler de l’actualité de Candide et des prochains événements auxquels il est possible de participer. Enfin, rappeler que Candide est toujours prêt à accueillir de nouveaux membres semble essentiel.

\sphinxAtStartPar
Il est tout à fait possible que grand nombre des visiteurs du site ne savent pas réellement ce qu’est Candide, une page de présentation sera alors très utile. Sur cette page, après une description plus détaillée de l’association, un résumé des objectifs atteint ainsi que les photos du comité au fil des ans, permettra d’humaniser le site et de permettre une vision d’ensemble sur l’activité de l’association.

\sphinxAtStartPar
Comme préalablement stipulé, Candide mise sur l’action, il est alors essentiel que le site internet compte un page ou chaque événement organisé est expliqué. Les associations, externes au Collège du Sud, avec lesquelles Candide a travaillé seront mises en avant sur cette page, car il est important que le comité mette en avant son historique d’actions pour montrer que ses actions sont concrètes.

\sphinxAtStartPar
Finalement une page contact n’apportera que du positif au site et au comité, car il mettra à disposition un moyen de joindre et de rejoindre l’association en tout temps.


\section{Les Maquettes}
\label{\detokenize{chapitre-02:les-maquettes}}

\subsection{n°1}
\label{\detokenize{chapitre-02:n1}}
\sphinxAtStartPar
À partir des information dont Candide m’a fait part, je me suis lancée dans la conception d’une première maquette. Celle\sphinxhyphen{}ci était directement basée sur le moodboard, afin de remplir au mieux les demandes de Candide. Je n’ai alors pas réellement ajouté d’éléments, j’ai simplement appliqué les demandes de Candide dans le but qu’elle correspondre exactement à leur souhaits.\\
Ce premier prototype a demandé plus de temps pour être créer, car il s’agissait de ma première utilisation de Adobe XD. En effet, bien que cet outil m’ait été inconnu jusqu’à ce moment, il semblait être parfait pour la création de maquette de site web. La compréhension de ce logiciel me pris un certain temps, mais la qualité des rendus en valait la peine.


\subsection{n°2}
\label{\detokenize{chapitre-02:n2}}
\sphinxAtStartPar
Après avoir terminé la première maquette, j’ai pris la décision d’en proposer deux nouvelles qui permettront peut\sphinxhyphen{}être à Candide d’avoir plus de choix et de voir autre chose que ce à quoi ils pouvaient s’attendre.

\sphinxAtStartPar
La deuxième maquette reste plutôt sobre dans ses couleurs, mais les polices apportent l’aspect plus fun du comité étudiant. Cependant la structure du site reste la même, seule certains aspects de la mise\sphinxhyphen{}en\sphinxhyphen{}page ont été modifiés. Avec ce nouveau prototype, la page de présentation des comité est beaucoup plus claire.\\
Ceoendant, les couleurs restent trop ternes et les membres de Candide aimerait donner à leur site un aspect moins grisatre, pour vraiment mettre en avant la bienveillance du comité.

\begin{figure}[htbp]
\centering
\capstart

\noindent\sphinxincludegraphics[width=1.000\linewidth]{{Proto2}.png}
\caption{2ème Prototype Teste}\label{\detokenize{chapitre-02:id2}}\end{figure}


\subsection{n°3}
\label{\detokenize{chapitre-02:n3}}
\sphinxAtStartPar
Le troisième et dernier prototype est lui beaucoup classe. Bien que les couleurs soient sombres, le vert domine et met en avant le souhait de Candide d’agir pour la préservation de l’environnement. Il donne aussi une image beaucoup plus professionel à Candide, mais ce n’est pas nécessairement ce que le comité veut mettre en avant. Il est effectivement important de garder en tête qu’il s’agit d’un comité étudiant et ne pas rendre ce site trop classique.

\begin{figure}[htbp]
\centering
\capstart

\noindent\sphinxincludegraphics[width=1.000\linewidth]{{Proto3}.png}
\caption{3ème Prototype Teste}\label{\detokenize{chapitre-02:id3}}\end{figure}


\section{Le Choix Final}
\label{\detokenize{chapitre-02:le-choix-final}}
\sphinxAtStartPar
Après plusieurs discussions et modifications de chaque maquette, l’idée choisie a été la première. Elle est celle qui correspondait d’ailleurs plus à la description de ce que Candide voulait, bien qu’elle ait subit beaucoup de modifications pour la rendre plus classe mais tout en restant jeune. Les changements apporté ont d’ailleurs beaucoup été indpiré du 3ème prototype dans le but d’apporter un dimension beaucoup plus professionelle au site.

\begin{figure}[htbp]
\centering
\capstart

\noindent\sphinxincludegraphics[width=1.000\linewidth]{{Proto_final}.png}
\caption{Maquette finale du site de Candide}\label{\detokenize{chapitre-02:id4}}\end{figure}


\chapter{Programmation}
\label{\detokenize{chapitre-03:programmation}}\label{\detokenize{chapitre-03::doc}}

\section{Apprentissage}
\label{\detokenize{chapitre-03:apprentissage}}
\sphinxAtStartPar
N’ayant que très peu de connaissances en programmation, j’ai dû, dès l’été, me familiariser avec les langages HTML et CSS afin de pouvoir coder mon site internet. Pour ce faire, j’ai d’abord créé quelques petits projet en suivant des tutos sur internet, puis j’ai essayé de m’en détacher pour tenter de faire un mini\sphinxhyphen{}site sans simplement suivre les gestes de quelqu’un. Ensuite, pour comprendre l’utilisation des classes, des grids et d’autres technologies de ce genre, je me suis tournée vers des jeux pédagogiques et des vidéos explicatives.


\section{Difficultés}
\label{\detokenize{chapitre-03:difficultes}}
\sphinxAtStartPar
Après avoir revu plusieurs fois la maquette avec le comité pour qu’elle réponde au mieux à leurs besoins et m’être renseignée au propos de ces nouvelles technologies qui allaient m’être utiles, Je me suis enfin lancée dans la programmation de ce site web.

\sphinxAtStartPar
Cette étape a été particulièrement difficile, car j’ai été confrontée à de nombreux défis que je devais surmonter de manière autonome. Bien qu’ayant passé beaucoup de temps à essayer de comprendre et utiliser correctement les différentes technologies, j’ai fait face à énormément de difficultés auxquelles je n’avais pas pensé. En effet, pour la réalisation de ce projet il m’a fallu résoudre une très grande quantité de problèmes inattendus.\\
J’ai notamment eu beaucoup de difficultés à organiser les différents layers. En effet, il n’était pas rare que des éléments se retrouve au très mauvais endroit et que je ne sache pas comment le déplacer. Un des problèmes qui est souvent revenu est un bloc blanc au bas de mes pages, j’ignore d’ailleurs encore la raison de sa présence. De plus, beaucoup de problèmes liés notamment à mon manque d’habitude sont apparus.


\section{Apprendre de ses erreurs}
\label{\detokenize{chapitre-03:apprendre-de-ses-erreurs}}
\sphinxAtStartPar
Cette partie du TM m’a permis de prendre en autonomie car elle m’a forcé à trouver seule des solutions à mes problèmes. On nous apprend souvent à aller voir la réponse si on fait une erreur, mais là il s’agissait d’aller voir le corriger et le comprendre pour l’appliquer dans un autre contexte. Cette manière de faire demande de comprendre exactement chaque mot de notre code, ce qui fait beaucoup, surtout pour une novice comme je le suis. Cette étape a alors été autant enrichissante qu’éprouvante !



\renewcommand{\indexname}{Index}
\printindex
\end{document}