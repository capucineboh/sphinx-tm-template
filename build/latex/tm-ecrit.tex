%% Generated by Sphinx.
\def\sphinxdocclass{report}
\documentclass[a4,10pt,french]{sphinxmanual}
\ifdefined\pdfpxdimen
   \let\sphinxpxdimen\pdfpxdimen\else\newdimen\sphinxpxdimen
\fi \sphinxpxdimen=.75bp\relax
\ifdefined\pdfimageresolution
    \pdfimageresolution= \numexpr \dimexpr1in\relax/\sphinxpxdimen\relax
\fi
%% let collapsible pdf bookmarks panel have high depth per default
\PassOptionsToPackage{bookmarksdepth=5}{hyperref}

\PassOptionsToPackage{warn}{textcomp}
\usepackage[utf8]{inputenc}
\ifdefined\DeclareUnicodeCharacter
% support both utf8 and utf8x syntaxes
  \ifdefined\DeclareUnicodeCharacterAsOptional
    \def\sphinxDUC#1{\DeclareUnicodeCharacter{"#1}}
  \else
    \let\sphinxDUC\DeclareUnicodeCharacter
  \fi
  \sphinxDUC{00A0}{\nobreakspace}
  \sphinxDUC{2500}{\sphinxunichar{2500}}
  \sphinxDUC{2502}{\sphinxunichar{2502}}
  \sphinxDUC{2514}{\sphinxunichar{2514}}
  \sphinxDUC{251C}{\sphinxunichar{251C}}
  \sphinxDUC{2572}{\textbackslash}
\fi
\usepackage{cmap}
\usepackage[T1]{fontenc}
\usepackage{amsmath,amssymb,amstext}
\usepackage[francais]{babel}


\usepackage{times}


\usepackage[Sonny]{fncychap}
\ChNameVar{\Large\normalfont\sffamily}
\ChTitleVar{\Large\normalfont\sffamily}
\usepackage{sphinx}

\fvset{fontsize=auto}
\usepackage{geometry}


% Include hyperref last.
\usepackage{hyperref}
% Fix anchor placement for figures with captions.
\usepackage{hypcap}% it must be loaded after hyperref.
% Set up styles of URL: it should be placed after hyperref.
\urlstyle{same}

\addto\captionsfrench{\renewcommand{\contentsname}{Table des matières}}

\usepackage{sphinxmessages}
\setcounter{tocdepth}{1}


%\usepackage[titles]{tocloft}
%\cftsetpnumwidth {1.25cm}\cftsetrmarg{1.5cm}
%\setlength{\cftchapnumwidth}{0.75cm}
%\setlength{\cftsecindent}{\cftchapnumwidth}
%\setlength{\cftsecnumwidth}{1.25cm}
\newcommand{\seminarytitle}{Développement Web}
\newcommand{\customizeinfos}{}


\title{Création du site web destiné au Comité Étudiant Humanitaire du Collège du Sud}
\date{mai 25, 2023}
\release{Version finale}
\author{Capucine Böhning}
\newcommand{\sphinxlogo}{\vbox{}}
\renewcommand{\releasename}{Collège du sud, Travail de maturité}
\makeindex
\begin{document}

\ifdefined\shorthandoff
  \ifnum\catcode`\=\string=\active\shorthandoff{=}\fi
  \ifnum\catcode`\"=\active\shorthandoff{"}\fi
\fi

\pagestyle{empty}
\sphinxmaketitle
\pagestyle{plain}
\sphinxtableofcontents
\pagestyle{normal}
\phantomsection\label{\detokenize{index::doc}}



\chapter{Introduction}
\label{\detokenize{introduction:introduction}}\label{\detokenize{introduction::doc}}

\section{Présentation du travail}
\label{\detokenize{introduction:presentation-du-travail}}
\sphinxAtStartPar
Le développement web est une discipline essentielle dans notre société actuelle, car il présente de nombreux avantages pour le développement d’activités commerciales ou associatives. Avoir un site web n’est plus simplement un atout, mais une nécessité pour les structures de toutes tailles. Bien qu’il existe de nombreux outils permettant de créer un site internet en quelques clics, cela n’est souvent pas suffisant. Pour véritablement se démarquer en ligne, il est crucial de proposer une interface utilisateur parfaite, où chaque détail est soigneusement pensé. C’est pourquoi de nombreuses entreprises préfèrent faire appel à des professionnels qualifiés. Malgré le coût que cela peut représenter, le retour sur investissement est rapide. En effet, une visibilité accrue permet d’augmenter rapidement l’audience et d’attirer de potentiels clients. De plus, un site web offre un gain de temps considérable pour la diffusion d’informations. Tous ces facteurs font des services professionnels de développement web une solution rentable.

\sphinxAtStartPar
Le développement web est un domaine vaste offrant de nombreuses possibilités, telles que la communication, le commerce, la création de réseaux ou encore la collecte de données. Dans ce travail de maturité, nous allons nous concentrer sur l’aspect communication en créant un site internet dédié au comité étudiant humanitaire du Collège du Sud. Il est évident que Candide a besoin d’un moyen de communication facilement accessible, permettant à un plus grand nombre de personnes de suivre les activités du groupe. Bien que l’idée de proposer une plateforme de commerce en ligne ou de solliciter des investisseurs ne concerne pas directement Candide, un site web est nécessaire pour faciliter le contact avec les différents partenaires avec lesquels le comité travaille et collecte des fonds.

\sphinxAtStartPar
Étant donné que le comité fait partie intégrante du Collège du Sud et n’a pas le statut d’association, il ne peut pas faire appel à des professionnels du développement web. En tant que membre de ce comité, l’auteur de ce travail a donc pris l’initiative de prendre en charge ce développement afin d’offrir à l’association le site vitrine dont elle a besoin.


\section{Objectifs du projet}
\label{\detokenize{introduction:objectifs-du-projet}}
\sphinxAtStartPar
Ce type de plateforme présente de nombreux avantages pour le groupe, notamment en termes de visibilité au sein du CSUD et à l’extérieur de l’établissement, ainsi que pour informer sur les événements à venir.


\subsection{Visibilité au sein de l’école}
\label{\detokenize{introduction:visibilite-au-sein-de-l-ecole}}
\sphinxAtStartPar
Bien que Candide existe au sein du Collège du Sud depuis de nombreuses années, ses activités restent encore trop peu connues. De nombreux étudiants, voire même des professeurs, ignorent encore les actions menées par le comité. Le site web vient combler ce manque en fournissant un accès rapide à l’information. De plus, son identité visuelle promeut une image jeune et dynamique du groupe, permettant aux utilisateurs de se faire une idée du comité dès leur arrivée sur le site.


\subsection{Visibilité à l’extérieur de l’école}
\label{\detokenize{introduction:visibilite-a-l-exterieur-de-l-ecole}}
\sphinxAtStartPar
Candide collabore régulièrement avec d’autres associations externes au Collège du Sud. Ces associations ont besoin d’un moyen simple pour découvrir Candide. Jusqu’à présent, les interactions se faisaient principalement par le biais de contacts et de rencontres. Le site permet donc d’atteindre un public beaucoup plus large et bien plus simplement. Les rapports inter\sphinxhyphen{}associations sont alors grandement facilités. De plus, ce site internet regroupe toutes les actions préalablement faites par le groupe. Ceci permet aussi aux partenaires  de découvrir le dynamisme de l’association et de commencer la collaboration en toute confiance.


\subsection{Informer des évènements à venir}
\label{\detokenize{introduction:informer-des-evenements-a-venir}}
\sphinxAtStartPar
Ce site joue un rôle essentiel dans la promotion et l’information des événements à venir. Il vise à motiver les étudiants à rejoindre le comité, à informer sur les actions récentes et, surtout, à mettre en avant les événements futurs. Étant un comité humanitaire, il accorde une grande importance aux relations humaines, à l’intégration de nouveaux membres et aux rencontres avec des étudiants externes lors des événements. Ainsi, le site met en avant le contact et les moments à partager pour favoriser ces interactions.


\section{Utilisateurs visés}
\label{\detokenize{introduction:utilisateurs-vises}}
\sphinxAtStartPar
Comme mentionné dans le chapitre précédent, deux types d’utilisateurs sont visés.
\begin{enumerate}
\sphinxsetlistlabels{\arabic}{enumi}{enumii}{}{.}%
\item {} 
\sphinxAtStartPar
\sphinxstylestrong{Les étudiants du Collège du Sud}: Candide faisant partie intégrante de la vie étudiante de l’établissement, il est essentiel que l’ensemble des élèves soient mis au courant de ses actions. Le comité doit alors pouvoir faire connaître les activités passées et surtout futures aux potentiels participants.

\item {} 
\sphinxAtStartPar
\sphinxstylestrong{Les partenaires externes}: Le comité base beaucoup de ses actions sur des collaborations. Ainsi, les associations humanitaires externes au CSUD sont évidemment visées par ce site internet. Il permet de déployer le champ d’action et de découvrir de nombreuses autres manières d’agir pour les causes que Candide souhaite défendre. Par ce biais, le contact est facilité et offre un aperçu direct sur les objectifs et ambitions du comité notamment dans le cadre de partenariats.

\end{enumerate}


\section{Structure du travail}
\label{\detokenize{introduction:structure-du-travail}}
\sphinxAtStartPar
Ce travail est séparé en six parties distinctes :
\begin{enumerate}
\sphinxsetlistlabels{\arabic}{enumi}{enumii}{}{.}%
\item {} 
\sphinxAtStartPar
La documentation « utilisateur » explique comment mettre en route le projet d’un point de vue des internautes qui consulteront le site. Ce premier axe présente aussi les différentes fonctionnalités du travail et ses apports.

\item {} 
\sphinxAtStartPar
La documentation « développeur » permet d’expliquer les différentes étapes de création du projet et sa structure générale. Aussi, ce volet présente les technologies utilisées tout au long de la création de ce site web.

\item {} 
\sphinxAtStartPar
Un regard critique sur le projet réalisé et un retour sur les objectifs non\sphinxhyphen{}atteints. Cette section présente les problèmes restants et les modifications qu’il serait préférable d’apporter avant la mise en ligne du site.

\item {} 
\sphinxAtStartPar
Une conclusion indique les objectifs initiaux remplis et résume les points majeurs de ce travail.

\item {} 
\sphinxAtStartPar
Une section finale propose des perspectives d’améliorations futures.

\item {} 
\sphinxAtStartPar
Les annexes regroupent les travaux complémentaires liés au projet, notamment les choix artistiques et les besoins spécifiques de l’association.

\end{enumerate}


\chapter{Documentation utilisateur}
\label{\detokenize{chapitre-01:documentation-utilisateur}}\label{\detokenize{chapitre-01::doc}}

\section{Mise en Route}
\label{\detokenize{chapitre-01:mise-en-route}}
\sphinxAtStartPar
Ce projet n’est pas encore accessible en ligne sur un serveur permanent. Pour y accéder, il est nécessaire de suivre les étapes décrites dans la documentation développeur en utilisant le site GitPod. Cependant, il est prévu que le site soit mis en ligne lors de la présentation orale du travail de maturité en mai 2023.


\section{Dépendances}
\label{\detokenize{chapitre-01:dependances}}
\sphinxAtStartPar
Ce projet ne contient aucune dépendance externe.


\section{Principales fonctionnalités du site}
\label{\detokenize{chapitre-01:principales-fonctionnalites-du-site}}
\sphinxAtStartPar
Ce projet comporte de nombreuses fonctionnalités :


\subsection{Vue d’ensemble}
\label{\detokenize{chapitre-01:vue-densemble}}
\sphinxAtStartPar
La page d’accueil présente Candide comme le comité étudiant humanitaire du Collège du Sud. Elle donne un aperçu rapide des actualités à venir et passées du groupe, ainsi que des informations importantes pour y participer. Cette page encourage également les étudiants intéressés à rejoindre le comité.


\subsection{Informations précises}
\label{\detokenize{chapitre-01:informations-precises}}\begin{enumerate}
\sphinxsetlistlabels{\arabic}{enumi}{enumii}{}{.}%
\item {} 
\sphinxAtStartPar
La page « Présentation » fournit une description plus précise du rôle de Candide dans la vie étudiante du Collège du Sud. Elle présente également les différents projets menés par le groupe au fil des ans et les associations externes avec lesquelles il a collaboré.

\item {} 
\sphinxAtStartPar
La section « Actions » décrit de manière détaillée les derniers projets réalisés par le comité. Cette page vise à expliquer les valeurs de Candide et son mode de fonctionnement lors de l’organisation d’événements ou de collaborations.

\item {} 
\sphinxAtStartPar
Chaque élément du site est accompagné d’une image pour illustrer le projet. Il est important pour Candide de montrer que son activité est concrète et basée sur l’action. Ces images permettent de démontrer l’engagement de tous les membres du comité !

\end{enumerate}


\subsection{Navigation intuitive}
\label{\detokenize{chapitre-01:navigation-intuitive}}
\sphinxAtStartPar
Le site web dispose d’un menu de navigation simple qui permet aux visiteurs de naviguer entre les différentes sections. Ce menu assure une structure claire du contenu du site rendant son utilisation intuitive.


\subsection{Fonctionnalité pratique}
\label{\detokenize{chapitre-01:fonctionnalite-pratique}}
\sphinxAtStartPar
Le design du site est « responsive », ce qui signifie qu’il a été conçu pour être utilisable sur tous les types d’appareils. Ainsi, il s’adapte automatiquement à la taille de l’écran, offrant une expérience optimale sur différentes tailles d’écran.

\sphinxAtStartPar



\chapter{Documentation développeur}
\label{\detokenize{chapitre-02:documentation-developpeur}}\label{\detokenize{chapitre-02::doc}}

\section{Mise en Route}
\label{\detokenize{chapitre-02:mise-en-route}}\begin{enumerate}
\sphinxsetlistlabels{\arabic}{enumi}{enumii}{}{.}%
\item {} 
\sphinxAtStartPar
Pour mettre en route le site internet, il nécessaire de se rendre sur \sphinxhref{https://github.com/capucineboh/site-candide}{ce lien} (https://github.com/capucineboh/site\sphinxhyphen{}candide)\sphinxstepexplicit %
\begin{footnote}[1]\phantomsection\label{\thesphinxscope.1}%
\sphinxAtStartFootnote
https://github.com/capucineboh/site\sphinxhyphen{}candide
%
\end{footnote} en passant par chrome, puis y ajouter la particule \sphinxcode{\sphinxupquote{gitpod.io/\#}} au début.

\item {} 
\sphinxAtStartPar
À la suite de cette manipulation, le projet s’ouvre dans GitPod, une plateforme de développement en ligne permettant de travailler sur les projets sans installer d’environnement de développement. Ceci permet d’éviter l’installation d’application de développement.

\item {} 
\sphinxAtStartPar
Une fois le projet ouvert dans GitPod, il faut le démarrer. Pour ce faire, il faut inscrire la commande \sphinxcode{\sphinxupquote{python \sphinxhyphen{}m http.server 8000}} dans le terminal, puis sélectionner le bouton « open in browser » de la fenêtre ouverte à l’occasion de la commande effectuée.

\end{enumerate}

\sphinxAtStartPar
Après avoir suivi ces étapes, une page s’ouvre dans votre navigateur où vous pouvez naviguer sur le site.


\section{Installation}
\label{\detokenize{chapitre-02:installation}}
\sphinxAtStartPar
Si l’installation locale du projet est souhaitée, il est nécessaire de procéder de la manière suivante :
\begin{enumerate}
\sphinxsetlistlabels{\arabic}{enumi}{enumii}{}{.}%
\item {} 
\sphinxAtStartPar
Il faut ouvrir un terminal sur l’ordinateur où l’installation locale est souhaitée.

\item {} 
\sphinxAtStartPar
Par le biais de la commande \sphinxcode{\sphinxupquote{git clone https://github.com/capucineboh/site\sphinxhyphen{}candide.git}}, il est possible de cloner le dépôt Git et cela créera une copie locale du dépôt sur l’ordinateur.

\item {} 
\sphinxAtStartPar
Il sera nécessaire de naviguer jusqu’au répertoire du projet à l’aide de la commande \sphinxcode{\sphinxupquote{cd site\sphinxhyphen{}candide}}.

\item {} 
\sphinxAtStartPar
Créez un environnement virtuel pour le projet. Vous pouvez le faire en utilisant la commande \sphinxcode{\sphinxupquote{python \sphinxhyphen{}m venv venv}}.

\item {} 
\sphinxAtStartPar
Activez l’environnement virtuel en utilisant la commande appropriée selon votre système d’exploitation : Windows : \sphinxcode{\sphinxupquote{venv\textbackslash{}Scripts\textbackslash{}activate}}, macOS/Linux : \sphinxcode{\sphinxupquote{source venv/bin/activate}}.

\item {} 
\sphinxAtStartPar
Une fois les dépendances installées, vous pouvez démarrer le serveur en utilisant la commande \sphinxcode{\sphinxupquote{python manage.py runserver}}.

\item {} 
\sphinxAtStartPar
Le serveur de développement démarrera et vous pourrez accéder au site en ouvrant votre navigateur et en entrant \sphinxhref{http://localhost:8000}{cette URL} (http://localhost:8000) \sphinxstepexplicit %
\begin{footnote}[3]\phantomsection\label{\thesphinxscope.3}%
\sphinxAtStartFootnote
https://github.com/git\sphinxhyphen{}guides/git\sphinxhyphen{}clone
%
\end{footnote}.

\end{enumerate}

\sphinxAtStartPar
De cette manière, le site peut être ouvert localement à n’importe quel moment.


\section{Contributions}
\label{\detokenize{chapitre-02:contributions}}
\sphinxAtStartPar
Il est possible pour tous développeurs d’apporter des modifications à ce projet. Ce site demande d’ailleurs à être mis à jour de manière régulière, afin qu’il reste actif et utile pour le comité. Plusieurs manipulations sont nécessaires à la contribution au site web de Candide.
\begin{enumerate}
\sphinxsetlistlabels{\arabic}{enumi}{enumii}{}{.}%
\item {} 
\sphinxAtStartPar
Il faut forker \sphinxhref{https://github.com/capucineboh/site-candide}{ce lien} (https://github.com/capucineboh/site\sphinxhyphen{}candide)\sphinxfootnotemark[1], en passant par chrome, puis y ajouter la particule \sphinxcode{\sphinxupquote{gitpod.io\#}} au début.\\
« Forker » signifie créer une copie indépendante sur son propre compte GitHub. Cette copie permet de travailler sur le code sans affecter l’original, ce qui est voulu ici.

\item {} 
\sphinxAtStartPar
Maintenant le fork effectué, il est nécessaire de clôner le dépôt forké sur GitPod. Ceci est faisable grâce à la commande \sphinxcode{\sphinxupquote{git clone https://github.com/capucineboh/site\sphinxhyphen{}candide.git}}, l’effet de celle\sphinxhyphen{}ci est de pouvoir récupérer l’entièreté des documents et fichiers d’un dépôt pour se l’approprier directement.

\item {} 
\sphinxAtStartPar
Ainsi, il faut créer une branche pour la nouvelle contribution en inscrivant la commande suivante dans le terminal : \sphinxcode{\sphinxupquote{git checkout \sphinxhyphen{}b {[}nom\sphinxhyphen{}de\sphinxhyphen{}la\sphinxhyphen{}nouvelle\sphinxhyphen{}fonctionnalité{]}}}. Grâce à cette nouvelle branche, il est possible d’apporter les modifications souhaitées sans altérer le projet initial.

\item {} 
\sphinxAtStartPar
Une fois tous les changements effectués, il faut enregistrer ces éditions grâce à la commande \sphinxcode{\sphinxupquote{git add .}} suivie de \sphinxcode{\sphinxupquote{git commit \sphinxhyphen{}m {[}nom du commit{]}}}, puis “pousser“ les nouvelles fonctionnalités avec \sphinxcode{\sphinxupquote{git push origin {[}nom\sphinxhyphen{}de\sphinxhyphen{}la\sphinxhyphen{}nouvelle\sphinxhyphen{}fonctionnalité{]}}}.

\item {} 
\sphinxAtStartPar
Dans le but que ces modifications soient ajoutées au projet initial, une dernière manipulation est nécessaire : effectuer une pull request. Il doit être accompagné d’une description détaillée de la contribution. Pour ce faire, il est nécessaire de cliquer sur le bouton « Compare \& pull resquest » disponible à côté de la nouvelle branche créée. De cette manière l’auteur majeur du site Web de Candide, peut choisir d’accepter ou de rejeter la demande et, au cas où elle serait acceptée, le nom du contributeur apparaîtra dans la liste des auteurs du projet.

\end{enumerate}


\section{Structure du code et rôle des différents fichiers}
\label{\detokenize{chapitre-02:structure-du-code-et-role-des-differents-fichiers}}

\subsection{HTML}
\label{\detokenize{chapitre-02:html}}
\sphinxAtStartPar
Le code du site web de Candide est structuré conformément aux règles d’usage. Il est composé de quatre fichiers HTML correspondant aux quatre pages du site. Ceux\sphinxhyphen{}ci forment le squelette du site internet, car le langage de programmation HTML est un langage de balisage, ce qui signifie qu’il permet d’organiser les éléments d’une page et. de les classer selon leur type. Ces quatre fichiers déterminent la façon dont les différents éléments de la page, tels que les titres, les paragraphes, les images, les liens, les boutons et les formulaires, sont organisés et disposés sur celle\sphinxhyphen{}ci. C’est à l’intérieur de ces différents types de balises que se trouvent également les textes et les images qui forment le contenu du site. On les trouve sous la forme suivante : \sphinxcode{\sphinxupquote{\textless{}img\textgreater{}\textasciigrave{}\textasciigrave{}\textless{}a\textgreater{}\textless{}/a\textgreater{}\textasciigrave{}\textasciigrave{}\textless{}p\textgreater{}\textless{}/p\textgreater{}\textasciigrave{}\textasciigrave{}\textless{}h1\textgreater{}\textless{}/h1\textgreater{}}}


\subsubsection{Métadonnées}
\label{\detokenize{chapitre-02:metadonnees}}
\sphinxAtStartPar
Les métadonnées sont des informations qui ne sont pas affichées directement sur le site web, mais qui sont stockées dans le code HTML de la page et sont utilisées par les moteurs de recherche pour comprendre le contenu de la page et pour effectuer des opérations de référencement. En effet, elles les aident à savoir à quelle position le site web doit figurer lorsque des recherches sont effectuées.

\sphinxAtStartPar
Les métadonnées sont communément définies dans l’entête du code HTML contenu dans les balises \sphinxcode{\sphinxupquote{head}}.
La balise \sphinxcode{\sphinxupquote{charset}} est l’une des plus commune. Il s’agit de celle définissant les jeux des caractères utilisés dans le site. Le charset de ce site est le UTF\sphinxhyphen{}8, un format d’encodage de caractères permettant de représenter tous les caractères de toutes les langues du monde. Cela permet également au navigateur de savoir comment interpréter les caractères spéciaux et les accents utilisés.
\sphinxcode{\sphinxupquote{Description}} est une autre balise essentielle à un site internet, puisqu’elle contient une courte description du contenu du site internet. Celle\sphinxhyphen{}ci s’affichera d’ailleurs directement en dessous du titre du site lors d’une recherche sur un moteur de recherche.\\
La métadonnée \sphinxcode{\sphinxupquote{content=“width=1200px , initial\sphinxhyphen{}scale=1.0“ }} est importante dans une page web, car elle permet de définir la largeur de la fenêtre d’affichage initiale. On s’assure alors que la page sera affichée de manière cohérente sur des écrans de taille différente. Aussi, l’attribut \sphinxcode{\sphinxupquote{initial\sphinxhyphen{}scale=1.0}} permet de définir le niveau de zoom initial pour la page web. Elle permet donc de s’assurer que la page s’affichera dans sa taille originale sans être agrandie ou réduite automatiquement par le navigateur. Ces deux paramètres sont particulièrement importants pour offrir une expérience utilisateur cohérente et agréable, sans avoir à faire défiler horizontalement ou à zoomer. Ceci permet donc une bonne responsivité du site.\\
Cette partie du code est également souvent complétée par la balise \sphinxcode{\sphinxupquote{title}}. Celle\sphinxhyphen{}ci contient le titre de la page, celui inscrit sur l’onglet du site web, une fois le site ouvert.


\subsection{CSS}
\label{\detokenize{chapitre-02:css}}
\sphinxAtStartPar
Le code de ce projet contient également un fichier CSS. Il est utilisé pour définir la présentation visuelle de la page web, plus communément appelé le « style » de la page. Il permet donc de préciser les différentes caractéristiques des éléments du HTML, comme les couleurs, les polices, les tailles et les dispositions.\\
Ce langage de programmation permet également la création de mises en page plus complexes grâce à des techniques telles que les types de positions, les \sphinxcode{\sphinxupquote{z\sphinxhyphen{}index}}, les \sphinxcode{\sphinxupquote{grid}} et les \sphinxcode{\sphinxupquote{flex\sphinxhyphen{}box}}, qui permettent une bonne organisation du contenu.


\subsubsection{Responsivity}
\label{\detokenize{chapitre-02:responsivity}}
\sphinxAtStartPar
Le document CSS de ce projet est séparé en deux parties distinctes : une pour les écrans de plus de 1000 pixels de largeur (ordinateurs), et la seconde pour les appareils dotés d’un plus petit écran (smartphones). Cette distinction est possible grâce à la fonctionnalité \sphinxcode{\sphinxupquote{@media screen and (min/max\sphinxhyphen{}width: 1000px)}} alors l’affichage sera différent sur ces deux types d’écrans. Cette fonctionnalité s’appelle le « responsive design ». L’organisation interne de ces sections est similaire. En effet, elles commencent toutes deux par des « class » générales, puis se précisent avec d’autres \sphinxcode{\sphinxupquote{class}} complémentaires. L’ordre dépend de la visite classique du site web selon la barre de navigation, en commençant par la page d’accueil et en terminant par la page contact.

\sphinxAtStartPar
Le \sphinxcode{\sphinxupquote{Flex\sphinxhyphen{}box}} susnommé s’est également montré très utile pour rendre le site responsive. La technologie CSS \sphinxcode{\sphinxupquote{Flex\sphinxhyphen{}box}} permet de créer des mises en page flexibles et adaptables. En effet, les élément sont reconnus comme des conteneurs et des éléments internes, ceci permet de pouvoir manier simplement les objets. Elle aide donc les développeurs à créer des mises en page facilement modifiables sans avoir à utiliser des outils plus complexe ou à devoirs utiliser des frameworks externes.\\
L’outil \sphinxcode{\sphinxupquote{Flex\sphinxhyphen{}box}} offre une grande flexibilité dans l’organisation des différents éléments d’une page en permettant de contrôler l’alignement, la taille et l’ordre des éléments à l’intérieur d’un conteneur (ici banner). Il est également possible de facilement spécifier comment les éléments à l’intérieur d’un conteneur doivent être positionnés et comment ils doivent s’adapter à différentes tailles d’écran. Par exemple, il est possible de spécifier que les éléments doivent être alignés verticalement et horizontalement au centre, ou que les éléments doivent s’adapter à la largeur disponible, etc.\\
L’un des avantages majeurs de \sphinxcode{\sphinxupquote{Flex\sphinxhyphen{}box}} est sa capacité à simplifier la création de designs responsives. Avant l’existence d’une telle technologie, les développeurs devaient utiliser des techniques moins directes comme les tableaux pour créer des mises en page responsives. Cependant, ces méthodes étaient souvent maladroites, difficiles à mettre en œuvre et engendraient d’un grand nombre de problèmes.
\begin{enumerate}
\sphinxsetlistlabels{\arabic}{enumi}{enumii}{}{.}%
\item {} 
\sphinxAtStartPar
En utilisant la propriété \sphinxcode{\sphinxupquote{flex\sphinxhyphen{}direction}}, il est possible de spécifier comment les éléments doivent se positionner en fonction de l’espace disponible.

\item {} 
\sphinxAtStartPar
La propriété \sphinxcode{\sphinxupquote{align\sphinxhyphen{}content}} spécifie comment les éléments doivent se comporter lorsqu’ils sont répartis sur plusieurs lignes et permet d’organiser l’alignement des éléments sur un. système d’axe en deux dimensions.

\item {} 
\sphinxAtStartPar
L’élément \sphinxcode{\sphinxupquote{grid}} en parallèle à Flex\sphinxhyphen{}box est également très utile. Il permet de diviser l’espace d’une page en une grille composée de lignes et de colonnes, et de placer les éléments HTML à l’intérieur de cette grille de manière précise et flexible.

\end{enumerate}


\section{Instructions pour tester le projet}
\label{\detokenize{chapitre-02:instructions-pour-tester-le-projet}}\begin{enumerate}
\sphinxsetlistlabels{\arabic}{enumi}{enumii}{}{.}%
\item {} 
\sphinxAtStartPar
Pour mettre en route le site internet, il nécessaire de se rendre sur \sphinxhref{https://github.com/capucineboh/site-candide}{ce lien} (https://github.com/capucineboh/site\sphinxhyphen{}candide)\sphinxfootnotemark[1]make tm en passant par chrome, puis y ajouter la particule \sphinxcode{\sphinxupquote{gitpod.io\#}} au début.

\item {} 
\sphinxAtStartPar
À la suite de cette manipulation, le projet s’ouvre dans GitPod, une plateforme de développement en ligne permettant de travailler sur les projets sans installer d’environnement de développement. Ceci permet la non\sphinxhyphen{}nécessité d’installer quelque application de développement.

\item {} 
\sphinxAtStartPar
Une fois le projet ouvert dans GitPod, il faut le démarrer. Pour ce faire, il faut inscrire la commande \sphinxcode{\sphinxupquote{python \sphinxhyphen{}m http.server 8000}} dans le terminal, puis sélectionner le bouton « open in browser » de la fenêtre ouverte à l’occasion de la commande effectuée.

\item {} 
\sphinxAtStartPar
Pour finir, une page où il est possible de naviguer sur le site internet crée dans le cadre d’un Travail de Maturité s’est ouverte.

\end{enumerate}


\section{Présentation des outils sous\sphinxhyphen{}jacents au projet}
\label{\detokenize{chapitre-02:presentation-des-outils-sous-jacents-au-projet}}

\subsection{Technologies spécifiques}
\label{\detokenize{chapitre-02:technologies-specifiques}}
\sphinxAtStartPar
Le logiciel Adobe XD a été utilisé pour l’élaboration des maquettes de ce projet. C’est un logiciel de prototypage d’interfaces de sites web et d’applications mobiles / de bureau permettant de créer des maquettes interactives pour tester l’expérience utilisateur au mieux.\\
Ce logiciel offre de nombreux outils pour faciliter le processus de conception comme : la possibilité de créer des grid, des calques et des modèles pour garder une certaine cohérence dans le design. Parmi les avantages d’Adobe XD, on peut citer sa fonctionnalité de partage de prototypes permettant de présenter le design à des collaborateurs pour obtenir des retours rapidement. Il offre aussi la possibilité de créer des prototypes interactifs qui peuvent être utilisés pour tester l’expérience utilisateur et recueillir des commentaires et critiques. Cela permet d’optimiser le design avant le développement du site et de s’assurer que l’expérience utilisateur est optimale.\\
Dans le cadre de la création du site web du comité étudiant humanitaire du Collège du Sud, Adobe XD est un outil fondamental dans la conception et le prototypage de l’interface, afin de s’assurer que le design réponde aux besoins de l’utilisateur et de Candide avant de commencer le développement du site. En effet, cette fonctionnalité est avantage un majeur de ce logiciel, puisque les longues discussions au sujet de l’interface sont souvent genèse de nombreuses modifications.\\
De plus, la première prise en main de ce logiciel, bien que nécessitant l’aide de plusieurs tutoriels et de la documentation en ligne d’Adobe XD, reste assez intuitive. Plusieurs raccourcis sont d’ailleurs similaires à d’autres logiciels de la suite Adobe, notamment à In Design. Il est alors très agréable de travailler avec ce logiciel et il est facile d’obtenir les bases très rapidement.


\subsection{Principes de conception / de programmation}
\label{\detokenize{chapitre-02:principes-de-conception-de-programmation}}
\sphinxAtStartPar
Plusieurs techniques et bonnes pratiques sont appliquées dans la création du site web de Candide, afin d’améliorer sa qualité, son efficacité et sa maintenabilité.\\
Tout d’abord, ce site set accessible à tous. Le design est pensé pour rendre la navigation intuitive et naturelle. De plus, chaque image contient un texte de remplacement, ce qui rend la navigation plus agréable pour les personnes malvoyantes. Aussi, les couleurs sont plutôt contrastées, ce qui permet une meilleure analyse des différents éléments de la page.\\
Ensuite, le design du site web est responsive. La mise en page s’adapte donc à tous type d’écrans. Il est important de concevoir un design utilisable sur différents supports. À l’aide de techniques telles que les \sphinxcode{\sphinxupquote{media queries}} et les \sphinxcode{\sphinxupquote{Flex\sphinxhyphen{}box}} le site s’adapte à toutes les tailles d’appareils.\\
Le CSS peut parfois être répétitif dans sa structure. Lorsque nous souhaitons styliser plusieurs éléments similaires, nous avons tendance à créer différentes classes, avec seulement quelques différences mineures entre elles. Cependant, afin d’éviter les répétitions excessives, il est judicieux d’adopter une approche plus efficace. Nous pouvons créer une classe générale pour regrouper ces éléments et ajouter des classes supplémentaires pour spécifier les détails stylistiques particuliers. Cette méthode nous permet d’éviter un fichier CSS trop long et complexe. En suivant ces bonnes pratiques de programmation, nous optimisons notre travail et améliorons l’efficacité de notre code.


\bigskip\hrule\bigskip



\chapter{Regard Critique}
\label{\detokenize{chapitre-03:regard-critique}}\label{\detokenize{chapitre-03::doc}}

\section{Pistes d’améliorations}
\label{\detokenize{chapitre-03:pistes-dameliorations}}
\sphinxAtStartPar
La création du site Web du Comité étudiant humanitaire du Collège du Sud est un projet mené dans le cadre d’un Travail de Maturité. Ainsi son objectif est avant tout l’apprentissage. C’est pourquoi chaque étape prend un temps non\sphinxhyphen{}négligeable et que plusieurs points demandent encore à être corrigés. Il est d’ailleurs prévu que les points d’améliorations identifiés seront résolus pour l’oral du moi de mai.
\begin{enumerate}
\sphinxsetlistlabels{\arabic}{enumi}{enumii}{}{.}%
\item {} 
\sphinxAtStartPar
Le premier est la nav\sphinxhyphen{}bar (barre de navigation). Certe elle semble tout à fait correcte lorsqu’on l’utilise sur un écran de bureau, mais dès que la surface de l’écran se réduit, elle n’est plus facilement utilisable. En effet, les boutons servant à accéder aux différentes pages du site Web deviennent trop petits et les inscriptions sont illisibles. Il serait alors nécessaire de transformer cette barre de navigation en bouton « menu ». Il rendrait le site bien plus accessible et permettrait une large amélioration de l’expérience utilisateur mobile.

\item {} 
\sphinxAtStartPar
Les media queries du CSS sont uniquement pensés pour les écrans de bureaux et les smartphone. Cependant, l’utilisation du site internet sur des tablettes ne doit pas être négligée. Il serait donc intéressant d’ajouter une troisième \sphinxcode{\sphinxupquote{media query}}, celle\sphinxhyphen{}ci se trouvant entre les deux déjà existantes. De plus, sur ce type de taille d’écran, l’image de fond de chaque titre ne se met pas bien en place. Un cadre blanc se forme autour d’elle. Il faudrait alors trouver une solution pour que cette image se mette correctement en place. Ainsi, le site Web de Candide serait plus agréable à visiter sur des appareils de ce format.

\item {} 
\sphinxAtStartPar
Toujours au niveau de la responsivité du design, il faudrait probablement utiliser la caractéristique \sphinxcode{\sphinxupquote{float}} pour les « chapter » du site, car des éléments ont effectivement tendance à se coincer en dessous des « chapter » rendant l’affichage désordonné et la lecture des différents éléments compliquée.

\item {} 
\sphinxAtStartPar
Finalement, le site Web du comité étudiant humanitaire du Collège du Sud, ayant été programmé sur Google chrome, ne s’adapte pas correctement aux autres navigateurs internet. Le « progressive enhancement » consiste à concevoir et à développer une base fonctionnelle solide pour un site web, en se concentrant sur les fonctionnalités essentielles. Ensuite, des améliorations et des fonctionnalités plus avancées sont ajoutées de manière progressive, en tenant compte des capacités du navigateur ou de l’appareil utilisé par l’utilisateur. Ainsi, tous les utilisateurs bénéficient d’une expérience de base, tandis que ceux disposant de technologies plus avancées peuvent profiter d’une expérience optimisée.

\end{enumerate}


\section{Objectifs initiaux non réalisés}
\label{\detokenize{chapitre-03:objectifs-initiaux-non-realises}}

\subsection{Objectif non remplis}
\label{\detokenize{chapitre-03:objectif-non-remplis}}
\sphinxAtStartPar
Les membres de Candide ne sont pas spécialisés en informatique. C’est pourquoi le moyen de changement du contenu du site doit être simple. Ainsi, un des objectifs fixés au début de ce Travail de Maturité est de créer une manière de changer les éléments du site simplement pour n’importe qui. Cependant, cette tâche, n’ayant pas été considérée comme étant une priorité immédiate, n’a pas été complétée au profit d’autres éléments.


\subsection{Solutions possibles}
\label{\detokenize{chapitre-03:solutions-possibles}}
\sphinxAtStartPar
Afin de créer ce moyen simple de modification pour les membres du comité, il est possible de passer par le biais d’un CMS en ligne. Un CMS est un système de gestion de contenu en ligne qui permet à un utilisateur de gérer les publications de contenu sur un site Web, sans avoir à connaître de langages de programmation. Il s’agit alors exactement de ce dont les membres de l’association ont besoin pour tenir ce site à jour. Ces services sont disponibles en open source. Cependant, cette adaptation n’a pas pu être effectuée dans le cadre de ce travail.


\section{Difficultés rencontrées}
\label{\detokenize{chapitre-03:difficultes-rencontrees}}
\sphinxAtStartPar
Les difficultés majeures de l’élaboration de ce projet résident dans l’apprentissage du logiciel Adobe XD, de GitHub et GitPod, ainsi que des langages de programmation et la correction de bugs.
\begin{enumerate}
\sphinxsetlistlabels{\arabic}{enumi}{enumii}{}{.}%
\item {} 
\sphinxAtStartPar
Dans le but de créer les maquettes du site, il faut s’initier à Adobe XD, un logiciel de prototypage d’interfaces web, les applications mobiles et les applications de bureau. Les maquettes prennent entre 3 h et 5 h à être faites, sans compter les nombreuses modifications apportées au cours de multiples discussions avec Candide.

\item {} 
\sphinxAtStartPar
Une fois la maquette finale obtenue, de nouvelles difficultés arrivent, mais celles\sphinxhyphen{}ci résident dans la programmation en elle\sphinxhyphen{}même. En effet, le développement web étant une discipline qui n’avait pas été abordée avant ce Travail de Maturité, l’apprentissage s’est fait pendant l’été et directement lors de l’implémentation du site. C’est pourquoi, de nombreux problèmes tels que des difficultés à mettre en avant certains éléments, à télécharger des polices, à utiliser les \sphinxcode{\sphinxupquote{Flex box}}, à placer correctement les éléments, à rendre le site responsive et tant d’autres voient le jour tout au long de sa programmation.

\item {} 
\sphinxAtStartPar
Un problème récurrent est rencontré en lien avec l’utilisation du CSS et le manque d’expérience dans l’utilisation de Git. À chaque nouvelle session de travail, il est nécessaire de créer un nouveau fichier CSS et de le lier à toutes les pages HTML, car les modifications apportées ne sont pas reflétées. Bien que la résolution de ce problème soit relativement simple une fois comprise, il n’a pas été possible de déterminer si cela est dû à un manque de connaissances pratiques ou à des problèmes inhérents à GitPod. Par conséquent, une solution définitive n’a pas été trouvée.

\end{enumerate}

\sphinxAtStartPar
Beaucoup de temps a été investi dans l’apprentissage et dans la résolution de bugs. Ce premier projet de développement web a alors été très enrichissant.


\section{Fonctionnalités supplémentaires}
\label{\detokenize{chapitre-03:fonctionnalites-supplementaires}}
\sphinxAtStartPar
L’activité du comité étudiant humanitaire du Collège du Sud réside autour de la collecte de dons pour des associations externes. Ainsi, il serait bon de développer une fonctionnalité permettant de faire des dons directement à partir du site, comme un formulaire à remplir afin de donner cinq, dix, ou même cent francs à la collecte en cours.\\
Cet ajout rendrait les dons beaucoup plus simples et rapides. Il n’y aurait plus besoin de se déplacer, de se souvenir de prendre de l’argent et tout se ferait en quelques clics. Il est remarquable que lorsque les dons sont plus faciles à réaliser, ils conduisent inévitablement à une augmentation du nombre de dons effectués. Il est effectivement souvent observé que la contrainte « cash » d’une collecte peut dissuader certaines personnes. De plus, ce biais permettrait d’atteindre une audience plus large et potentiellement extérieure au Collège du Sud.\\
Aussi, dans l’air digitale dans laquelle nous vivons, il ne faut pas négliger le fait que de plus en plus de collectes se feront uniquement en ligne. Cette fonctionnalité moderniserait alors aussi l’organisation du comité et son image en soi.


\chapter{Conclusion}
\label{\detokenize{chapitre-04:conclusion}}\label{\detokenize{chapitre-04::doc}}
\sphinxAtStartPar
Les points majeurs de ce Travail de Maturité sont la création du design et la programmation du site Web en soi.
Le design du site répond aux besoins du comité étudiant humanitaire du Collège du Sud. Il permet de présenter le groupe, son activité et offre un moyen simple de le contacter. Ces objectifs de communication ont été remplis par le biais des différentes pages du site. De plus, un objectif majeur était de rendre un projet représentatif de Candide et les couleurs, les images et les formes du site sont pensées pour mettre en avant les valeurs du comité et donner une image fidèle aux membres qui le forment.\\
L’implémentation est également une partie majeure de la création de ce site Web, puisqu’il s’agit de transformer la maquette en code. Cette partie représente le plus gros défi de ce projet. En effet, la programmation a pris un temps conséquent, car cette section inclut également l’apprentissage des différents langages de programmation et de divers technologies complémentaires.


\chapter{Travaux futurs}
\label{\detokenize{chapitre-05:travaux-futurs}}\label{\detokenize{chapitre-05::doc}}
\sphinxAtStartPar
En raison de contraintes de temps, certains éléments n’ont pas pu être inclus dans le projet. Il serait donc pertinent de les ajouter prochainement.
\begin{enumerate}
\sphinxsetlistlabels{\arabic}{enumi}{enumii}{}{.}%
\item {} 
\sphinxAtStartPar
Les recherches légales liées au développement web n’ayant pas été faites, des éléments sont manquant. Il est effectivement obligatoire que les sites web comportent, en bas de chaque page, un bandeau où résident les mentions légales. Il est important que ces informations résident sur un site internet, car elles permettent aux utilisateurs de savoir s’il s’agit d’un site toujours actif ou non et d’y faire figurer des informations telles que les contacts, les adresses, les dates de mise\sphinxhyphen{}à\sphinxhyphen{}jour récentes, les noms des développeurs et contributeurs, …

\item {} 
\sphinxAtStartPar
De plus, afin d’améliorer l’expérience utilisateur, il serait intéressant de créer des liens cliquables pour directement accéder au compte Instagram de Candide ou envoyer un mail. Ces fonctionnalités permettraient de faciliter le contact et d’inciter les étudiants à rejoindre le groupe plus simplement.

\end{enumerate}

\appendix



\chapter{Annexes}
\label{\detokenize{chapitre-06:annexes}}\label{\detokenize{chapitre-06::doc}}

\section{Déclaration personnelle}
\label{\detokenize{chapitre-06:declaration-personnelle}}
{

\large
\begin{tabbing}
\textbf{Nom:} ~~~~~ \=  \= \studentlastname \\
\textbf{Prénom:}~~~ \>  \> \studentfirstname \\
\textbf{Adresse:}~~ \>  \> \studentaddress
\end{tabbing} 
}

\begin{normalsize}
\begin{enumerate}
\sphinxsetlistlabels{\arabic}{enumi}{enumii}{}{.}%
\item {} 
\sphinxAtStartPar
Je certifie que le travail Création du site web destiné au Comité Étudiant Humanitaire du Collège du Sud a été réalisé par moi conformément au Guide de travail des collèges et aux Lignes directrices de la DICS concernant la réalisation du Travail de maturité.

\item {} 
\sphinxAtStartPar
Je prends connaissance que mon travail sera soumis à une vérification de la
mention correcte et complète de ses sources, au moyen d’un logiciel de
détection de plagiat. Pour assurer ma protection, ce logiciel sera également
utilisé pour comparer mon travail avec des travaux écrits remis
ultérieurement, afin d’éviter des copies et de protéger mon droit d’auteur.
En cas de soupçon d’atteintes à mon droit d’auteur, je donne mon accord à la
direction de l’école pour l’utilisation de mon travail comme moyen de
preuve.

\item {} 
\sphinxAtStartPar
Je m’engage à ne pas rendre public mon travail avant l’évaluation finale.

\item {} 
\sphinxAtStartPar
Je m’engage à respecter la Procédure d’archivage des travaux de
maturité/travaux interdisciplinaires centrés sur un projet/travaux
personnels/travaux de maturité spécialisée (TM/TIP/TP/TMS) en vigueur dans
mon école.

\item {} 
\sphinxAtStartPar
J’autorise la consultation de mon travail par des tierces personnes à des
fins pédagogiques et/ou d’information interne à l’école :

{
\square \quad \text{oui} \\
\square \quad \text{non (car il contient des données personnelles et sensibles).}
}

\end{enumerate}

{
\large
\begin{tabbing}
\textbf{Lieu, date:} ~~ \=  \= \ldots\ldots\ldots\ldots\ldots\ldots\ldots\ldots\ldots\ldots\ldots\ldots\ldots\ldots\ldots\ldots\ldots \\
\textbf{Signature:} ~~~ \>  \> \ldots\ldots\ldots\ldots\ldots\ldots\ldots\ldots\ldots\ldots\ldots\ldots\ldots\ldots\ldots\ldots\ldots
\end{tabbing} 
}


\end{normalsize}


\section{Travaux complémentaires}
\label{\detokenize{chapitre-06:travaux-complementaires}}

\subsection{Recherches Artistiques}
\label{\detokenize{chapitre-06:recherches-artistiques}}
\sphinxAtStartPar
Dans le but de créer un design web cohérent, il est primordial de commencer par se pencher sur l’identité visuelle et l’univers de Candide. Après plusieurs discussions, certains aspects importants sont ressortis.\\
Tout d’abord, il s’agit d’un comité étudiant humanitaire, il est alors important que les couleurs et les images évoquent la bienveillance, la convivialité et la motivation. Les couleurs doivent alors être plutôt lumineuses et intenses afin de faire ressortir ces aspects. Aussi, la protection de l’environnement représente une valeur importante et majeure pour le comité, la prédominance du vert semble alors naturelle.\\
Ensuite, une atmosphère à l’inspiration rétro évoque la notion de recyclage et récupération. Une inspiration des 70’s induit aussi l’idée du « Peace and Love » ce qui représente plutôt bien la philosophie du comité. Cette inspiration se ressent par la couleur orange représentative de la période art\sphinxhyphen{}déco. Elle permet également d’énergiser les pages et de contraster avec le vert, dominant.\\
Les formes, elles aussi, ont leur importance. En effet, elles font partie du langage graphique et influent l’atmosphère du design. Ici, les formes rectangulaires sont adoucies, sauf pour les images qui elles gardent un format standard. L’aspect circulaire vient du logo préexistant, qui suit la forme du C et d’une mappemonde.\\
Les couleurs de ce site sont très vives, ce qui peut constituer une surcharge visuelle. C’est pourquoi une image plus douce sert d’arrière\sphinxhyphen{}plan en haut de chaque page du site Web. Cette image correspond parfaitement à l’identité visuelle de Candide et du site en règle générale : une prédominance de vert, des touches d’orange et la présence de fleurs qui rappelle l’importance de la protection de la nature pour le comité.

\begin{figure}[htbp]
\centering
\capstart

\noindent\sphinxincludegraphics[width=1.000\linewidth]{{moodboard}.jpg}
\caption{MoodBoard du comité étudiant humanitaire du Collège du Sud}\label{\detokenize{chapitre-06:id1}}\end{figure}


\subsection{Maquettes}
\label{\detokenize{chapitre-06:maquettes}}
\sphinxAtStartPar
Trois prototypes du site Web de Candide sont réalisés dans le cadre de ce Travail de Maturité et sont sujets à de nombreuses discussions afin d’arriver à un choix final. Chacun a connu de multiples modifications dans le but d’être fidèle à l’image du comité et à celle que le comité décide de véhiculer.


\subsubsection{Première maquette}
\label{\detokenize{chapitre-06:premiere-maquette}}
\sphinxAtStartPar
À partir des informations dont Candide a fait part, démarre la conception d’une première maquette. Celle\sphinxhyphen{}ci est directement basée sur le moodboard et cherche à être au plus proche de leurs souhaits.

\sphinxAtStartPar
Suite à cela, deux autres maquettes plus libres sont inventées, afin d’explorer des nouvelles pistes.


\subsubsection{Première maquette « d’exploration »}
\label{\detokenize{chapitre-06:premiere-maquette-dexploration}}
\sphinxAtStartPar
La première maquette « d’exploration » est plutôt sobre dans ses couleurs. Cependant, les polices apportent l’aspect plus décontracté du comité étudiant. La structure du site, quant à elle, reste la même et seuls certains éléments de la mise en page ont été modifiés. Avec ce nouveau prototype, la page de présentation des comités au fil des années est beaucoup plus claire.\\
Malheureusement, les couleurs restent trop ternes et les membres de Candide aimeraient donner à leur site un aspect moins grisâtre, pour mettre en avant la bienveillance du comité de façon plus marquée.

\begin{figure}[htbp]
\centering
\capstart

\noindent\sphinxincludegraphics[width=1.000\linewidth]{{Proto2}.png}
\caption{Première maquette « d’exploration »}\label{\detokenize{chapitre-06:id2}}\end{figure}


\subsubsection{Deuxième maquette « d’exploration »}
\label{\detokenize{chapitre-06:deuxieme-maquette-dexploration}}
\sphinxAtStartPar
La seconde maquette « d’exploration » amène un aspect plus moderne et classique que la précédente. Un vert sombre domine et met en avant le souhait de Candide d’agir pour la préservation de l’environnement. Le côté atténué de la couleur donne, quant à lui, une image beaucoup plus professionnelle au comité.  En revanche, après discussion avec les étudiants, ils ne désirent pas mettre cela en avant. Il est effectivement important de garder en tête qu’il s’agit d’un comité étudiant et ne pas rendre ce site trop classique.

\begin{figure}[htbp]
\centering
\capstart

\noindent\sphinxincludegraphics[width=1.000\linewidth]{{Proto3}.png}
\caption{Deuxième maquette « d’exploration »}\label{\detokenize{chapitre-06:id3}}\end{figure}


\subsubsection{Choix final}
\label{\detokenize{chapitre-06:choix-final}}
\sphinxAtStartPar
Après plusieurs discussions et modifications de chaque maquette, l’idée sélectionnée est la première. Elle est celle qui correspond d’ailleurs le plus auxsouhaits de Candide. Elle a subi beaucoup de modifications pour la rendre plus sobre et moderne. Les changements apportés sont d’ailleurs largement inspirés du troisième prototype dans le but d’apporter une dimension beaucoup plus professionnelle au site.

\begin{figure}[htbp]
\centering
\capstart

\noindent\sphinxincludegraphics[width=1.000\linewidth]{{Proto_final}.png}
\caption{Maquette finale du site de Candide}\label{\detokenize{chapitre-06:id4}}\end{figure}



\renewcommand{\indexname}{Index}
\printindex
\end{document}