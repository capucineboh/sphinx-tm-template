%% Generated by Sphinx.
\def\sphinxdocclass{report}
\documentclass[a4,10pt,french]{sphinxmanual}
\ifdefined\pdfpxdimen
   \let\sphinxpxdimen\pdfpxdimen\else\newdimen\sphinxpxdimen
\fi \sphinxpxdimen=.75bp\relax
\ifdefined\pdfimageresolution
    \pdfimageresolution= \numexpr \dimexpr1in\relax/\sphinxpxdimen\relax
\fi
%% let collapsible pdf bookmarks panel have high depth per default
\PassOptionsToPackage{bookmarksdepth=5}{hyperref}

\PassOptionsToPackage{warn}{textcomp}
\usepackage[utf8]{inputenc}
\ifdefined\DeclareUnicodeCharacter
% support both utf8 and utf8x syntaxes
  \ifdefined\DeclareUnicodeCharacterAsOptional
    \def\sphinxDUC#1{\DeclareUnicodeCharacter{"#1}}
  \else
    \let\sphinxDUC\DeclareUnicodeCharacter
  \fi
  \sphinxDUC{00A0}{\nobreakspace}
  \sphinxDUC{2500}{\sphinxunichar{2500}}
  \sphinxDUC{2502}{\sphinxunichar{2502}}
  \sphinxDUC{2514}{\sphinxunichar{2514}}
  \sphinxDUC{251C}{\sphinxunichar{251C}}
  \sphinxDUC{2572}{\textbackslash}
\fi
\usepackage{cmap}
\usepackage[T1]{fontenc}
\usepackage{amsmath,amssymb,amstext}
\usepackage[francais]{babel}


\usepackage{times}


\usepackage[Sonny]{fncychap}
\ChNameVar{\Large\normalfont\sffamily}
\ChTitleVar{\Large\normalfont\sffamily}
\usepackage{sphinx}

\fvset{fontsize=auto}
\usepackage{geometry}


% Include hyperref last.
\usepackage{hyperref}
% Fix anchor placement for figures with captions.
\usepackage{hypcap}% it must be loaded after hyperref.
% Set up styles of URL: it should be placed after hyperref.
\urlstyle{same}

\addto\captionsfrench{\renewcommand{\contentsname}{Table des matières}}

\usepackage{sphinxmessages}
\setcounter{tocdepth}{1}


%\usepackage[titles]{tocloft}
%\cftsetpnumwidth {1.25cm}\cftsetrmarg{1.5cm}
%\setlength{\cftchapnumwidth}{0.75cm}
%\setlength{\cftsecindent}{\cftchapnumwidth}
%\setlength{\cftsecnumwidth}{1.25cm}
\newcommand{\seminarytitle}{Développement Web}
\newcommand{\customizeinfos}{}


\title{Création du site internet destiné au Comité Étudiant Humanitaire du Collège du Sud}
\date{avr. 02, 2023}
\release{Version finale}
\author{Capucine Böhning}
\newcommand{\sphinxlogo}{\vbox{}}
\renewcommand{\releasename}{Collège du sud, Travail de maturité}
\makeindex
\begin{document}

\ifdefined\shorthandoff
  \ifnum\catcode`\=\string=\active\shorthandoff{=}\fi
  \ifnum\catcode`\"=\active\shorthandoff{"}\fi
\fi

\pagestyle{empty}
\sphinxmaketitle
\pagestyle{plain}
\sphinxtableofcontents
\pagestyle{normal}
\phantomsection\label{\detokenize{index::doc}}



\chapter{Introduction}
\label{\detokenize{introduction:introduction}}\label{\detokenize{introduction::doc}}

\section{Présentation du travail}
\label{\detokenize{introduction:presentation-du-travail}}
\sphinxAtStartPar
Une discipline ayant une place importante dans notre société actuelle est le développement web, car il présente de nombreux avantages pour développer une activité commerciale ou associative. En effet, avoir un site web n’est plus un atout, c’est un besoin pour les structures de toutes tailles. Il est vrai, qu’aujourd’hui, beaucoup d’outils existent pour créer un site internet en quelques clics, mais ceci est loin d’être suffisant. Car pour se faire une réelle bonne image en ligne, il faut que l’interface utilisateur soit parfaite et qu’aucun détail ne soit laissé au hasard. Pour cette raison, il est plus intéressant pour les entreprises de faire appel à des personnes qualifiées. Bien que cela représente un certain coût, le retour sur investissement se voit vite. Effectivement, la visibilité gagnée augmente rapidement l’audience et les clients potentiels. De plus, elle est aussi un gain de temps pour la diffusion d’informations ; tous ces facteurs pèsent dans la balance et rendent les services professionnels de programmations rentables.\\
Le développement web est un sujet vaste qui peut être exploité de nombreuses manières différentes.\\
Dans ce TM, nous allons aborder l’axe de la communication, par le biais de la création d’un site internet destiné au comité étudiant humanitaire du Collège du Sud. Il a effectivement été remarqué que Candide nécessite d’un moyen de communication facilement accessible et permettant à un plus grand nombre de personnes de suivre l’activité du groupe. Bien que l’idée de proposer une forme de commerce en ligne, ou de vouloir faire appel à des investisseurs potentiels ne touche absolument pas Candide ; un site web leur est nécessaire pour le contact avec les différents partenaires pour lesquels ils travaillent et récoltent des fonds.\\
Faisant partie intégrante du Collège du Sud et n’ayant pas le statut d’association, eux ne peuvent pas faire appel à des professionnels du développement web. Car en effet, ne faisant aucun bénéfice, l’investissement dans la création d’un site internet ne serait que pertes. Leurs projets étant d’une grande importance aux yeux de la jeunesse actuelle, l’initiative de leur créer ce site dont ils ont besoin a été prise. C’est pourquoi, il a été décidé que ce projet se portera sur l’association en elle\sphinxhyphen{}même.


\section{Objectifs du projet}
\label{\detokenize{introduction:objectifs-du-projet}}
\sphinxAtStartPar
Les nombreux avantages de ce genre de plateforme permettent d’atteindre des objectifs importants pour le groupe.


\subsection{La visibilité au sein de l’école}
\label{\detokenize{introduction:la-visibilite-au-sein-de-lecole}}
\sphinxAtStartPar
Bien que Candide existe au sein du Collège de Sud depuis de nombreuses années, son activité reste encore trop peu connue. Il est vrai que beaucoup d’étudiants et même de professeurs continuent à ignorer les actions du comité. Ce site répond alors à ce manque de connaissances en permettant l’accès rapide à l’information. Aussi, son identité visuelle promeut une image jeune et dynamique du groupe, ce qui permet aux utilisateurs du site de rapidement se faire une idée sur le comité.


\subsection{La visibilité hors de l’école}
\label{\detokenize{introduction:la-visibilite-hors-de-lecole}}
\sphinxAtStartPar
Candide collabore très fréquemment avec d’autres associations. Celles\sphinxhyphen{}ci, extérieures aux Collège du Sud, ont besoin d’un moyen simple de connaître Candide. Jusqu’à présent, le groupe fonctionnait par le biais de contacts et de rencontres. Mais maintenant, le site permet d’atteindre un public beaucoup plus large et bien plus simplement. Les rapports inter\sphinxhyphen{}associations sont alors grandement facilité. De plus, ce site internet regroupe toutes les actions préalablement faites par le groupe, ceci permet aussi aux partenaires potentiels de se sentir plus vite en confiance pour collaborer.


\subsection{Informer des évènements à venir}
\label{\detokenize{introduction:informer-des-evenements-a-venir}}
\sphinxAtStartPar
Ce site est un outil de communication pour promouvoir et informer des évènements à venir. Motiver des étudiants à rejoindre le comité, prévenir des récentes actions et surtout des prochains événements sont alors mis sur le devant de la scène. Candide est un comité humanitaire, et qui privilégie donc les relations humaines. Un point d’orgue est alors marqué lorsqu’il s’agit d’intégrer de nouvelles personnes au groupe et de rencontrer des étudiants externes au comité lors d’événements. Le contact et les moments à partager sont alors mis en avant dans ce site afin de les favoriser.


\section{Utilisateurs visés}
\label{\detokenize{introduction:utilisateurs-vises}}
\sphinxAtStartPar
Comme mentionné dans le chapitre précédent, deux types d’utilisateurs sont visés.
\begin{enumerate}
\sphinxsetlistlabels{\arabic}{enumi}{enumii}{}{.}%
\item {} 
\sphinxAtStartPar
Les étudiants du Collège du Sud. Candide faisant partie intégrante de la vie étudiante de l’établissement, il est important que l’ensemble des élèves soient mis au courant de ses actions. Il est alors important pour le comité de pouvoir faire connaître les activités passées et surtout futures aux potentiels participants.

\item {} 
\sphinxAtStartPar
Le comité base beaucoup de ses actions sur des collaborations. Ainsi, les associations humanitaires externes au CSUD sont évidemment visées par ce site internet. Il permet de déployer le champ d’action et de découvrir de nombreuses autres manière d’agir pour les causes que Candide souhaite défendre. Par ce biais, le contact est facilité et offre un aperçu direct sur les objectifs et ambitions du groupe étudiant dans le cadre de partenariat et même en règle générale.

\end{enumerate}


\section{Structure du travail}
\label{\detokenize{introduction:structure-du-travail}}
\sphinxAtStartPar
Ce travail est séparé en six parties distinctes :
\begin{enumerate}
\sphinxsetlistlabels{\arabic}{enumi}{enumii}{}{.}%
\item {} 
\sphinxAtStartPar
Section de documentation présentant comment mettre en route le projet d’un point de vue utilisateur. Ce premier axe présente aussi les différentes fonctionnalités du travail et ses apports.

\item {} 
\sphinxAtStartPar
Documentation ”développeur” permet d’expliquer les différentes étapes de création du projet et sa structure générale. Aussi, ce volet présente les technologies utilisées tout au long de la création de ce site web.

\item {} 
\sphinxAtStartPar
Regard critique sur le projet réalisé et présente un retour sur les objectifs non\sphinxhyphen{}atteints. En effet, elle présente les problèmes restant et les modifications qu’il serait préférable d’apporter avant la mise en ligne du site.

\item {} 
\sphinxAtStartPar
Conclusion par l’indication des objectifs initiaux remplis et un résumé des points majeurs de ce travail.

\item {} 
\sphinxAtStartPar
Sections finales offrant une ouverture sur les potentiels travaux futurs qui n’ont pu être implémentés pour diverses raisons. Ce dernier volet est suivi des annexes, notamment composé des travaux complémentaires au projet.

\end{enumerate}

\sphinxAtStartPar



\chapter{Documentation utilisateur}
\label{\detokenize{chapitre-01:documentation-utilisateur}}\label{\detokenize{chapitre-01::doc}}

\section{Mise en Route}
\label{\detokenize{chapitre-01:mise-en-route}}
\sphinxAtStartPar
Ce projet n’est pas encore disponible en ligne sur un serveur permanent, il est alors encore obligatoire de passer par le site du GitPod pour y accéder, comme expliqué dans la documentation développeur. Cependant, il est prévu que le site soit mis en ligne lors de la présentation orale du travail de maturité en mai 2023.


\section{Dépendances}
\label{\detokenize{chapitre-01:dependances}}
\sphinxAtStartPar
Ce projet ne contient aucune dépendance externe.


\section{Principales fonctionnalités du site}
\label{\detokenize{chapitre-01:principales-fonctionnalites-du-site}}
\sphinxAtStartPar
Ce projet comporte de nombreuses fonctionnalités :


\subsection{Vue d’ensemble}
\label{\detokenize{chapitre-01:vue-densemble}}
\sphinxAtStartPar
Une page d’accueil présente Candide comme le comité étudiant humanitaire du Collège du Sud. Elle permet également un aperçu rapide sur l’actualité future et récente du groupe, ainsi que les informations importantes pour y participer. Cette page invite aussi les étudiants intéressés à rejoindre le comité.


\subsection{Vue détaillée}
\label{\detokenize{chapitre-01:vue-detaillee}}\begin{enumerate}
\sphinxsetlistlabels{\arabic}{enumi}{enumii}{}{.}%
\item {} 
\sphinxAtStartPar
La page nommée « Présentation » définit plus précisément le rôle de Candide dans la vie étudiante du Collège du Sud. Cette page expose aussi les différents projets menés par le groupe sur les dernières années et les noms des associations externes avec qui il a collaboré.

\item {} 
\sphinxAtStartPar
La section « actions » permet une description détaillée des derniers projets menés par le comité. Cette page a pour objectif de préciser quelles sont les valeurs de Candide et son mode de fonctionnement lors d’organisation d’évènements ou de collaborations.

\item {} 
\sphinxAtStartPar
Chaque élément de ce site est accompagné d’une image dans le but d’illustrer le projet. Il est important pour Candide de montrer que son activité est concrète et se base sur l’action, alors la présence de photos se justifie évidemment.

\end{enumerate}


\subsection{Vue raccourcie}
\label{\detokenize{chapitre-01:vue-raccourcie}}
\sphinxAtStartPar
Aussi, le site web comprend un menu de navigation simple permettant aux visiteurs de naviguer entre les différentes sections. Ce menu permet aussi une structure claire du contenu du site.


\subsection{Fonctionnalité pratique}
\label{\detokenize{chapitre-01:fonctionnalite-pratique}}
\sphinxAtStartPar
Le design du site est responsive. Ce qui signifie qu’il a été pensé pour être utilisable sur tous type d’appareils. Ainsi, il s’adapte automatiquement à la taille de l’écran, permettant une expérience optimale sur différentes tailles d’écran.


\chapter{Documentation développeur}
\label{\detokenize{chapitre-02:documentation-developpeur}}\label{\detokenize{chapitre-02::doc}}

\section{Mise en Route}
\label{\detokenize{chapitre-02:mise-en-route}}
\sphinxAtStartPar
Pour mettre en route le site internet, il nécessaire de se rendre sur le lien suivant en passant par chrome : gitpod.io\#/https://github.com/capucineboh/site\sphinxhyphen{}candide .\\
À la suite de cette manipulation, le projet s’ouvre dans GitPod, une plateforme de développement en ligne permettant de travailler sur les projets sans installer d’environnement de développement. Ceci permet la non\sphinxhyphen{}nécessité d’installer quelque application de développement.\\
Une fois le projet ouvert dans GitPod, il faut le démarrer. Pour ce faire, il faut inscrire la commande ”python \sphinxhyphen{}m http.server 8000” dans le terminal, puis sélectionner le bouton « open in browser » de la fenêtre ouverte à l’occasion de la commande effectuée.\\
Pour finir, une page où il est possible de naviguer sur le site s’est ouverte.


\section{Installation}
\label{\detokenize{chapitre-02:installation}}
\sphinxAtStartPar
Source Documentation GitHub

\sphinxAtStartPar
S’il est souhaité d’installer le projet localement, il est nécessaire de procéder de la manière suivante :
\begin{enumerate}
\sphinxsetlistlabels{\arabic}{enumi}{enumii}{}{.}%
\item {} 
\sphinxAtStartPar
Il faut ouvrir un terminal sur l’ordinateur où l’installation locale est souhaitée.

\item {} 
\sphinxAtStartPar
Par le biais de la commande ”git clone https://github.com/capucineboh/site\sphinxhyphen{}candide.git”, il est rendu possible de cloner le dépôt Git et cela créera une copie locale du dépôt sur l’ordinateur.

\item {} 
\sphinxAtStartPar
Il sera nécessaire de naviguer jusqu’au répertoire du projet à l’aide de la commande ”cd site\sphinxhyphen{}candide”.

\item {} 
\sphinxAtStartPar
Il est important de vérifier si npm est installé sur l’ordinateur en inscrivant ”npm \sphinxhyphen{}v” dans terminal local. Ainsi, le serveur de développement peut être démarré par la commande ”npm start” et ouvrir automatiquement le site chargé dans votre navigateur web. Pour arrêter le serveur, il suffit d’entrer Ctrl+C dans le terminal.

\end{enumerate}

\sphinxAtStartPar
De cette manière, le site peut être ouvert à n’importe quel moment, directement depuis l’ordinateur en question.


\section{Contributions}
\label{\detokenize{chapitre-02:contributions}}
\sphinxAtStartPar
Source https://github.com/git\sphinxhyphen{}guides/git\sphinxhyphen{}clone
Il est possible pour tous développeur d’apporter des modifications à ce projet, elles sont d’ailleurs les bienvenues. Ce site demande d’ailleurs à être mis\sphinxhyphen{}à\sphinxhyphen{}jour de manière régulière, afin qu’il reste actif et utile pour le comité. Plusieurs manipulations sont nécessaires à la contribution au site web de Candide.
\begin{enumerate}
\sphinxsetlistlabels{\arabic}{enumi}{enumii}{}{.}%
\item {} 
\sphinxAtStartPar
Il faut forker le dépôt Git disponible sous ce lien : https://github.com/capucineboh/site\sphinxhyphen{}candide.git. « Forker » signifie créer une copie indépendante sur son propre compte GitHub. Cette copie vous permet de travailler sur le code sans affecter l’original, ce qui est voulu ici.

\item {} 
\sphinxAtStartPar
Maintenant le fork effectué il est nécessaire de clôner le dépôt forké sur GitPod. Ceci est faisable grâce à la commande ”git clone https://github.com/capucineboh/site\sphinxhyphen{}candide.git”, l’effet de celle\sphinxhyphen{}ci est de pouvoir récupérer l’entièreté des documents et fichiers d’un dépôt pour se l’approprier directement. Ainsi, il faut créer une branche pour la nouvelle contribution en inscrivant la commande suivante dans le terminal : « git checkout \sphinxhyphen{}b {[}nom\sphinxhyphen{}de\sphinxhyphen{}la\sphinxhyphen{}nouvelle\sphinxhyphen{}fonctionnalité{]} ». Cette nouvelle branche créée, il est rendu possible d’apporter les modifications souhaitées sans altéré le projet initial.

\item {} 
\sphinxAtStartPar
Une fois tous les changements effectués, il faut enregistrer ces éditions grâce aux commandes « git add . » suivie de « git commit \sphinxhyphen{}m {[}nom du commit{]} », puis push les nouvelles fonctionnalités avec « git push origin {[}nom\sphinxhyphen{}de\sphinxhyphen{}la\sphinxhyphen{}nouvelle\sphinxhyphen{}fonctionnalité{]} ».

\item {} 
\sphinxAtStartPar
Dans le but que ces modifications soient ajoutées au projet initial, une dernière manipulation est nécessaire. Celle\sphinxhyphen{}ci est d’effectuer une pull request, elle doit être accompagnée d’une description détailler de la contribution. Pour ce faire, il est nécessaire de cliquer sur le bouton « Compare \& pull resquest » disponible à côté de la nouvelle branche créée. De cette manière l’auteur majeur du site Web de Candide, peut choisir d’accepter ou de rejeter la demande et, au cas où elle serait acceptée, le nom de contributeur apparaîtra dans la liste des auteurs du projet.

\end{enumerate}


\section{Structure du code et rôle des différents fichiers}
\label{\detokenize{chapitre-02:structure-du-code-et-role-des-differents-fichiers}}

\subsection{Structure du code}
\label{\detokenize{chapitre-02:structure-du-code}}

\subsubsection{HTML}
\label{\detokenize{chapitre-02:html}}
\sphinxAtStartPar
Source https://developer.mozilla.org/fr/docs/Web/HTML/Element/meta
Le code du site web de Candide est structuré conformément aux règles d’usage. Il est composé de quatre fichiers HTML correspondant aux quatre pages du site. Ceux\sphinxhyphen{}ci forment le squelette du site internet, car le langage de programmation HTML est un langage de balisage, ce qui signifie qu’il permet d’organiser les éléments d’une page par leur type. Ces quatre fichiers déterminent la façon dont les différents éléments de la page, tels que les titres, les paragraphes, les images, les liens, les boutons et les formulaires, sont organisés et disposés sur celle\sphinxhyphen{}ci. C’est à l’intérieur de ces différents types de balises que se trouvent également les textes et les images qui forment le contenu du site.


\paragraph{Métadonnées}
\label{\detokenize{chapitre-02:metadonnees}}
\sphinxAtStartPar
Les métadonnées sont des informations qui ne sont pas affichées directement sur le site web, mais qui sont stockées dans le code HTML de la page et sont utilisées par les moteurs de recherches pour comprendre le contenu de la page et pour effectuer des opérations de référencement. En effet, elles aident les navigateurs à savoir à quelle position le site web doit figurer lorsque des recherches sont effectuées.

\sphinxAtStartPar
Les métadonnées sont communément définies dans l’entête de code HTML contenue dans les balises .
La balise ”charset” est l’une des plus commune. Il s’agit de celle définissant les jeux des caractères utilisés dans le site. Le charset de ce site est le UTF\sphinxhyphen{}8, un format d’encodage de caractères permettant de représenter tous les caractères de toutes les langues du monde. Cela permet également au navigateur de savoir comment interpréter les caractères spéciaux et les accents utilisés, pour les afficher correctement aux utilisateurs.
”Description“ est une autre balise essentielle à un site internet, puisqu’elle contient une courte description du contenu du site internet. Celle\sphinxhyphen{}ci s’affichera d’ailleurs directement en dessous du titre du site lors d’une recherche internet.\\
La métadonnée ” content=’width=1200px , initial\sphinxhyphen{}scale=1.0’ ” est importante dans une page web, car elle permet de définir la largeur de la fenêtre d’affichage initiale. On s’assure alors que la page sera affichée de manière cohérente sur des écrans de taille différente. Aussi, l’attribut ”initial\sphinxhyphen{}scale=1.0” permet de définir le niveau de zoom initial pour la page web. Elle permet donc de s’assurer que la page s’affichera dans sa taille originale sans être agrandie ou réduite automatiquement par le navigateur. Ces deux paramètres sont particulièrement importants pour offrir une expérience utilisateur cohérente et agréable, sans avoir à faire défiler horizontalement ou à zoomer. Ceci permet donc une bonne responsivité du site.\\
Cette partie du code est également souvent complétée par la balise . Celle\sphinxhyphen{}ci contient le titre de la page, celui inscrit sur l’onglet du site web, une fois le site ouvert.


\subsubsection{CSS}
\label{\detokenize{chapitre-02:css}}
\sphinxAtStartPar
Le code de ce projet contient également un fichier CSS. Il est utilisé pour définir la présentation visuelle de la page web, plus communément appelé le « style » de la page. Il permet donc de préciser les différentes caractéristiques des éléments du HTML, comme les couleurs, les polices, les tailles et les dispositions.\\
Le langage de programmation CSS permet également la création de mises en page plus complexes grâce à des techniques telles que les types de positions, les « z\sphinxhyphen{}index », les « grid » et les « Flex\sphinxhyphen{}box », qui permettent une bonne organisation du contenu.


\paragraph{Responsivity}
\label{\detokenize{chapitre-02:responsivity}}
\sphinxAtStartPar
Le document CSS de ce projet est séparé en deux parties distinctes : une pour les écrans de plus de 1000 pixels de largeur (ordinateurs), et la seconde pour les appareils dotés d’un plus petit écran (smartphones). Cette distinction est possible grâce à la fonctionnalité « @media screen and (min/max\sphinxhyphen{}width: 1000px) », alors l’affichage sera différent sur ces deux types d’écrans différents. Cette fonctionnalité s’appelle le « responsive design ». L’organisation interne de ces sections est similaire. En effet, elles commencent toutes deux par des « class » générales, puis se précisent avec d’autres « class » complémentaires. L’ordre dépend de l’ordre chronologique de la visite du site web, en commençant par la page d’accueil et en terminant par la page contact.

\sphinxAtStartPar
Le Flex\sphinxhyphen{}box susnommé s’est également montré très utile pour rendre le site responsive. La technologie CSS Flex\sphinxhyphen{}box permet de créer des mises\sphinxhyphen{}en\sphinxhyphen{}page flexibles et adaptables. Elle aide les développeurs à créer des mises en page facilement modifiables sans avoir à utiliser des outils plus complexe ou à devoirs utiliser des frameworks externes.\\
L’outil Flex\sphinxhyphen{}box offre une grande flexibilité dans l’organisation des différents éléments d’une page en permettant de contrôler l’alignement, la taille et l’ordre des éléments à l’intérieur d’un conteneur (ici banner). Il est également possible de facilement spécifier comment les éléments à l’intérieur d’un conteneur doivent être positionnés et comment ils doivent s’adapter à différentes tailles d’écran. Par exemple, il est possible de spécifier que les éléments doivent être alignés verticalement et horizontalement au centre, ou que les éléments doivent s’adapter à la largeur disponible, …\\
L’un des avantages majeurs des Flex\sphinxhyphen{}box est sa capacité à simplifier la création de designs responsives. Avant l’existence d’une telle technologie, les développeurs devaient utiliser des techniques moins directes comme les tableaux pour créer des mises en page responsives. Cependant, ces méthodes étaient souvent maladroites, difficiles à mettre en œuvre et engendraient d’un grand nombre de problèmes.\\
Avec les Flex\sphinxhyphen{}box, la création d’une mise en page responsive est grandement simplifiée. Les éléments peuvent être disposés en rangées ou en colonnes et le conteneur flex s’adapte automatiquement à la largeur de l’écran.
\begin{enumerate}
\sphinxsetlistlabels{\arabic}{enumi}{enumii}{}{.}%
\item {} 
\sphinxAtStartPar
En utilisant la propriété « flex\sphinxhyphen{}direction », il est possible de spécifier comment les éléments doivent se positionner en fonction de l’espace disponible.

\item {} 
\sphinxAtStartPar
La propriété « align\sphinxhyphen{}content » spécifie comment les éléments doivent se comporter lorsqu’ils sont répartis sur plusieurs lignes et permet d’organiser l’alignement des éléments sur l’axe vertical.

\item {} 
\sphinxAtStartPar
L’élément “grid“ en parallèle à Flex\sphinxhyphen{}box est également très utile.

\end{enumerate}


\section{Instructions pour tester le projet}
\label{\detokenize{chapitre-02:instructions-pour-tester-le-projet}}
\sphinxAtStartPar
Pour mettre en route le site internet, il nécessaire de se rendre sur le lien suivant en passant par chrome: gitpod.io\#/https://github.com/capucineboh/site\sphinxhyphen{}candide .\\
À la suite de cette manipulation, le projet s’ouvre dans GitPod, une plateforme de développement en ligne permettant de travailler sur les projets sans installer d’environnement de développement. Ceci permet la non\sphinxhyphen{}nécessité d’installer quelque application de développement.\\
Une fois le projet ouvert dans GitPod, il faut le démarrer. Pour ce faire, il faut inscrire la commande ”python \sphinxhyphen{}m http.server 8000” dans le terminal, puis sélectionner le bouton « open in browser » de la fenêtre ouverte à l’occasion de la commande effectuée.\\
Pour finir, une page où il est possible de naviguer sur le site internet crée dans le cadre d’un Travail de Maturité s’est ouverte.


\section{Présentation des outils sous\sphinxhyphen{}jacents au projet}
\label{\detokenize{chapitre-02:presentation-des-outils-sous-jacents-au-projet}}
\sphinxAtStartPar
\#\#\#Technologies spécifiques

\sphinxAtStartPar
Le logiciel Adobe XD a été utilisé pour l’élaboration des maquettes de ce projet. C’est un logiciel de prototypage d’interfaces de sites web et d’applications mobiles / de bureau permettant de créer des maquettes interactives pour tester l’expérience utilisateur au mieux.\\
Ce logiciel offre de nombreux outils pour faciliter le processus de conception comme : la possibilité de créer des grid, des layout et des modèles pour garder une certaine cohérence dans le design. Parmi les avantages d’Adobe XD, on peut citer sa fonctionnalité de partage de prototypes permettant de présenter le design à des collaborateurs pour obtenir des retours rapidement. Il offre aussi la possibilité de créer des prototypes interactifs qui peuvent être utilisés pour tester l’expérience utilisateur et recueillir des commentaires et critiques. Cela permet d’optimiser le design avant le développement du site et de s’assurer que l’expérience utilisateur est optimale.\\
Dans le cadre de la création du site web du comité étudiant humanitaire du Collège du Sud, Adobe XD est un outil très utile dans la conception et le prototypage de l’interface, dans le but de s’assurer que le design réponde aux besoins de l’utilisateur et de Candide avant de commencer le développement du site. En effet, cette fonctionnalité est un majeur de ce logiciel, puisque les longues discussions au sujet de l’interface sont souvent genèse de nombreuses modifications.\\
De plus, la première prise en main de ce logiciel, bien que nécessitant l’aide de plusieurs tutoriels et de la documentation en ligne d’Adobe XD, reste assez intuitive. Plusieurs raccourcis sont d’ailleurs similaires à d’autres logiciels de la suite Adobe, notamment à In Design. Il est alors très agréable de travailler avec ce logiciel et il est facile d’obtenir les bases très rapidement.


\subsection{Principes de conception / de programmation}
\label{\detokenize{chapitre-02:principes-de-conception-de-programmation}}
\sphinxAtStartPar
Plusieurs techniques et bonnes pratiques sont appliquées dans la création du site web de Candide, afin d’améliorer sa qualité, son efficacité et sa maintenabilité.\\
Tout d’abord, ce site est accessible à tous. Le design est pensé pour rendre la navigation intuitive et naturelle. De plus, chaque image contient un texte de remplacement, qui rend leur présence possible pour les personnes malvoyantes. Aussi, les couleurs sont plutôt contrastées, ce qui permet une meilleure analyse des différents éléments de la page.\\
Ensuite, le design du site web est responsive. La mise\sphinxhyphen{}en\sphinxhyphen{}page s’adapte donc à tous type d’écrans. Il est important de concevoir un design utilisable sur différents supports. À l’aide de techniques telles que les media queries et les Flex\sphinxhyphen{}box le site correspond à toutes les tailles d’appareils.\\
Aussi, le CSS pouvant être très répétitif, il est défini par groupes. En effet, les classes de certains éléments de mêmes types ne se différencient que par quelques caractéristiques, il est alors intéressant de créer une classe générale et que les précisions stylistiques soient apportées dans d’autres classes. Ce biais permet d’éviter les nombreuses répétitions qui peuvent souvent être trouvées dans un fichier CSS lorsque les bonnes pratiques de programmations ne sont pas appliquées.


\chapter{Regard Critique}
\label{\detokenize{chapitre-03:regard-critique}}\label{\detokenize{chapitre-03::doc}}

\section{Pistes d’améliorations}
\label{\detokenize{chapitre-03:pistes-dameliorations}}
\sphinxAtStartPar
La création du site Web du Comité étudiant humanitaire du Collège du Sud est un projet mené dans le cadre d’un Travail de Maturité, ainsi son objectif est avant tout l’apprentissage. C’est pourquoi chaque étape prend un temps non\sphinxhyphen{}négligeable et que plusieurs points demandent encore à être corrigés. Il est d’ailleurs prévu que les points d’améliorations identifiés seront résolus pour la version orale.
\begin{enumerate}
\sphinxsetlistlabels{\arabic}{enumi}{enumii}{}{.}%
\item {} 
\sphinxAtStartPar
Le premier est la nav\sphinxhyphen{}bar. Certes elle semble tout à fait correcte lorsqu’on l’utilise sur un écran de bureau, mais dès que la surface de l’écran se réduit, elle n’est plus facilement utilisable. En effet, les boutons servant à accéder aux différentes pages du site Web deviennent trop petits et les inscriptions sont illisibles. Il serait alors nécessaire d’apporter cette modification. Elle rendrait le site bien plus accessible et permettrait une large amélioration de l’expérience utilisateur mobile.

\item {} 
\sphinxAtStartPar
Les media queries du CSS sont uniquement pensées pour les écrans de bureaux et les smartphone. Cependant, l’utilisation du site internet sur des tablettes de doit par être négligée. Il serait donc intéressant d’ajouter une troisième media query, celle\sphinxhyphen{}ci se trouvant entre les deux déjà existante. De plus, sur ce type de taille d’écran, l’image de fond de chaque entête ne se met pas bien en place. Un cadre blanc se forme autour d’elle. Il faudrait alors trouver une solution pour que cette image se mette correctement en place.  Ainsi, le site Web de Candide serait plus agréable à visiter sur des appareils de ce format.

\item {} 
\sphinxAtStartPar
Toujours au niveau de la responsivité du design, il faudrait probablement utiliser la caractéristique ”float” pour les ”chapter” du site. Car des éléments ont effectivement tendance à se coincer en dessous des ”chapter” rendant l’affichage désordonné et la lecture des différents éléments compliquée.

\item {} 
\sphinxAtStartPar
Finalement, le site Web du comité étudiant humanitaire du Collège du Sud, ayant été programmé sur Google chrome, ne s’adapte pas correctement aux autres navigateurs de recherches. Le ”progressive enhancement” aurait pu être une approche utilisée dans le but de permettre à tous les utilisateurs, indépendamment de leur navigateur ou de leur appareil, de profiter d’une bonne expérience lors de la consultation du site Web.

\end{enumerate}


\section{Éléments ne répondant pas aux objectifs initiaux fixés}
\label{\detokenize{chapitre-03:elements-ne-repondant-pas-aux-objectifs-initiaux-fixes}}

\subsection{Objectif non remplit}
\label{\detokenize{chapitre-03:objectif-non-remplit}}
\sphinxAtStartPar
Les membres de Candide ne sont pas spécialisés en informatique. C’est pourquoi le moyen de changement du contenu du site doit être simple. Ainsi, un des objectifs fixés au début de ce Travail de Maturité est de créer une manière de changer les éléments du site simplement pour n’importe qui. Cependant, cette tâche, n’ayant pas été considérée comme étant une priorité immédiate, n’a pas été complétée au profit d’autres éléments.


\subsection{Pistes de solutions}
\label{\detokenize{chapitre-03:pistes-de-solutions}}
\sphinxAtStartPar
Afin de créer ce moyen simple de modification pour les membres du comité, il est possible de passer par le biais d’un CMS en ligne. Un CMS est un système de gestion de contenu en ligne qui permet à un utilisateur de gérer les publications de contenu sur un site Web, sans avoir à connaître de langages de programmation. Il s’agit alors exactement de ce dont les membres de l’association ont besoin pour tenir ce site à jour. Ces services sont disponibles en open source. Cependant, l’adaptation d’un tel code n’a pas pu être effectuée dans le cadre de ce travail.


\section{Difficultés rencontrées}
\label{\detokenize{chapitre-03:difficultes-rencontrees}}
\sphinxAtStartPar
Les difficultés majeures de l’élaboration de ce projet résident dans l’apprentissage du logiciel Adobe XD, de GitHub et GitHub, ainsi que des langages de programmation ainsi que dans la correction de bugs.
\begin{enumerate}
\sphinxsetlistlabels{\arabic}{enumi}{enumii}{}{.}%
\item {} 
\sphinxAtStartPar
Dans le but de créer les maquettes du site, il faut s’initier à Adobe XD, un logiciel de prototypage d’interfaces pour les sites web, les applications mobiles et les applications de bureau. Les maquettes prennent entre 3 h et 5 h à être faites, sans compter les nombreuses modifications apportées au cours de multiples discussions avec Candide.

\item {} 
\sphinxAtStartPar
Une fois la maquette finale obtenue, de nouvelles difficultés arrivent, mais celles\sphinxhyphen{}ci résident dans la programmation en elle\sphinxhyphen{}même. En effet, le développement web étant une discipline qui n’avait pas été abordée avant ce Travail de Maturité, l’apprentissage s’est fait pendant l’été et directement lors de l’implémentation du site. C’est pourquoi, de nombreux problèmes tels que des difficultés à mettre en avant certains éléments, à télécharger des polices, à utiliser les flex box, à placer correctement les éléments, à rendre le site responsive et tant d’autres voient le jour tout au long de sa programmation.

\item {} 
\sphinxAtStartPar
Un problème majeur revenant à de multiples reprises est lié au CSS et au manque d’expérience en utilisation de Git. Effectivement, à chaque nouvelle session de travail, il est nécessaire de créer un nouveau fichier CSS et de le lier à nouveau aux différentes pages HTML, car les modifications apportées ne s’affichent pas. Ceci représente une contrainte importante, bien qu’une fois connue, elle soit simple à résoudre.
Il n’a pas été identifié si ce problème venait du manque de connaissances pratiques ou de GitPod en lui\sphinxhyphen{}même, il n’a alors jamais pu être résolu de manière définitive.

\end{enumerate}

\sphinxAtStartPar
Beaucoup de temps a été investi dans l’apprentissage et dans la résolution de bugs, ce premier projet de développement web a alors été très enrichissant.


\section{Fonctionnalités supplémentaires}
\label{\detokenize{chapitre-03:fonctionnalites-supplementaires}}
\sphinxAtStartPar
L’activité du comité étudiant humanitaire du Collège du Sud réside autour de la collecte de dons pour des associations externes. Ainsi, il serait bon de développer une fonctionnalité permettant de faire des dons directement à partir du site. Comme un formulaire à remplir afin de donner cinq, dix, ou même cent francs à la collecte en cours.\\
Cet ajout rendrait les dons beaucoup plus simples et rapides, il n’y aurait plus besoin de se déplacer, de se souvenir de prendre de l’argent et tout se ferait en quelques clics. Si les dons se font plus simple, force est de constater qu’ils se font aussi plus nombreux. Il est effectivement souvent observé que la contrainte ”cash” d’une collecte peut dissuader certaines personnes. De plus, ce biais permettrait d’atteindre une audience plus large et potentiellement extérieure au Collège du Sud.\\
Aussi, dans l’air digitale dans laquelle nous vivons, il ne faut pas négliger le fait que de plus en plus de collecte se feront uniquement en ligne. Cette fonctionnalité moderniserait alors aussi l’organisation du comité et son image en soi.


\chapter{Conclusion}
\label{\detokenize{chapitre-04:conclusion}}\label{\detokenize{chapitre-04::doc}}
\sphinxAtStartPar
Les points majeurs de ce Travail de Maturité sont la création du design et la programmation du site Web en soi.
Le design du site répond aux besoins du comité étudiant humanitaire du Collège du Sud. Il permet de présenter le groupe, son activité et offre un moyen simple de le contacter. Ces objectifs ont été remplis par le biais des différentes pages du site. De plus, un objectif majeur était de rendre un projet représentatif de Candide et les couleurs, les images et les formes du site sont pensées pour mettre en avant les valeurs du comité et donner une image fidèle aux membres qui le forment.\\
L’implémentation est également une partie majeure de la création de ce site Web, puisqu’il s’agit de transformer la maquette en code. Cette partie représente le plus gros défi de ce projet. En effet, la programmation a pris un temps conséquent, car cette section inclut également l’apprentissage des différents langages de programmation et de divers technologies complémentaires.


\chapter{Travaux futurs}
\label{\detokenize{chapitre-05:travaux-futurs}}\label{\detokenize{chapitre-05::doc}}
\sphinxAtStartPar
Certains éléments n’ont pas pu être intégrés au projet par manque de temps, c’est pourquoi il serait intéressant de les ajouter dans un futur proche :
\begin{enumerate}
\sphinxsetlistlabels{\arabic}{enumi}{enumii}{}{.}%
\item {} 
\sphinxAtStartPar
Les recherches légales liées au développement web n’ayant pas été faites, des éléments sont manquant. Il est effectivement obligatoire que les sites web comportent, en bas de chaque page, un bandeau où résident les mentions légales. Il est important que ces informations résident sur un site internet, car elles permettent aux utilisateurs de savoir s’il s’agit d’un site toujours actif ou non et d’y faire figurer des informations telles que les contacts, les adresses, les dates de mise\sphinxhyphen{}à\sphinxhyphen{}jour récentes, les noms des développeurs et contributeurs, …

\item {} 
\sphinxAtStartPar
De plus, afin d’améliorer l’expérience utilisateur, il serait intéressant de créer des liens cliquables pour directement accéder au compte Instagram de Candide ou envoyer un mail. Ces fonctionnalités permettraient de faciliter le contact et d’inciter les étudiants à rejoindre le groupe plus simplement.

\end{enumerate}

\appendix



\chapter{Déclaration personnelle}
\label{\detokenize{chapitre-06:declaration-personnelle}}\label{\detokenize{chapitre-06::doc}}
{
\setstretch{1.5}
\large
\begin{tabbing}
\textbf{Nom:} ~~~~~ \=  \= \studentlastname \\
\textbf{Prénom:}~~~ \>  \> \studentfirstname \\
\textbf{Adresse:}~~ \>  \> \studentaddress
\end{tabbing} 
}

\begin{normalsize}
\begin{enumerate}
\sphinxsetlistlabels{\arabic}{enumi}{enumii}{}{.}%
\item {} 
\sphinxAtStartPar
Je certifie que le travail

{\large \textbf{\tmtitle} }

\sphinxAtStartPar
a été réalisé par moi conformément au Guide de travail des collèges et aux
Lignes directrices de la DICS concernant la réalisation du Travail de
maturité.

\item {} 
\sphinxAtStartPar
Je prends connaissance que mon travail sera soumis à une vérification de la
mention correcte et complète de ses sources, au moyen d’un logiciel de
détection de plagiat. Pour assurer ma protection, ce logiciel sera également
utilisé pour comparer mon travail avec des travaux écrits remis
ultérieurement, afin d’éviter des copies et de protéger mon droit d’auteur.
En cas de soupçon d’atteintes à mon droit d’auteur, je donne mon accord à la
direction de l’école pour l’utilisation de mon travail comme moyen de
preuve.

\item {} 
\sphinxAtStartPar
Je m’engage à ne pas rendre public mon travail avant l’évaluation finale.

\item {} 
\sphinxAtStartPar
Je m’engage à respecter la Procédure d’archivage des travaux de
maturité/travaux interdisciplinaires centrés sur un projet/travaux
personnels/travaux de maturité spécialisée (TM/TIP/TP/TMS) en vigueur dans
mon école.

\item {} 
\sphinxAtStartPar
J’autorise la consultation de mon travail par des tierces personnes à des
fins pédagogiques et/ou d’information interne à l’école :

{
\setstretch{1.5}
\square \quad \text{oui} \\
\square \quad \text{non (car il contient des données personnelles et sensibles).}
}

\end{enumerate}

{
\setstretch{2.5}
\large
\begin{tabbing}
\textbf{Lieu, date:} ~~ \=  \= \ldots\ldots\ldots\ldots\ldots\ldots\ldots\ldots\ldots\ldots\ldots\ldots\ldots\ldots\ldots\ldots\ldots \\
\textbf{Signature:} ~~~ \>  \> \ldots\ldots\ldots\ldots\ldots\ldots\ldots\ldots\ldots\ldots\ldots\ldots\ldots\ldots\ldots\ldots\ldots
\end{tabbing} 
}


\end{normalsize}


\section{Travaux complémentaires}
\label{\detokenize{chapitre-06:travaux-complementaires}}

\subsection{Recherches Artistiques}
\label{\detokenize{chapitre-06:recherches-artistiques}}
\sphinxAtStartPar
Dans le but de créer un design web cohérent, il est primordial de commencer par se pencher sur l’identité visuelle et l’univers de Candide. Après plusieurs discussions, certains aspects importants sont ressortis.\\
Tout d’abord, il s’agit d’un comité étudiant humanitaire, il est alors important que les couleurs et les images évoquent la bienveillance, la convivialité et la motivation. Les couleurs doivent alors être plutôt lumineuses et intenses afin de faire ressortir ces aspects. Aussi, la protection de l’environnement représente une valeur importante et majeure pour le comité, la prédominance du vert semble alors naturelle.\\
Ensuite, une atmosphère à l’inspiration rétro évoque la notion de recyclage et récupération. Une inspiration des 70’s induit aussi l’idée du « Peace and Love » ce qui représente plutôt bien la philosophie du comité. Cette inspiration se ressent par la présence d’orange, couleur représentative de la période art\sphinxhyphen{}déco. Elle permet également d’énergiser les pages et de contraster avec le vert, dominant.\\
Les formes, elles aussi, ont leur importance. En effet, elles font partie du langage graphique et influent l’atmosphère du design. Ici, les formes rectangulaires sont adoucies, sauf pour les images qui elles gardent un format standard. L’aspect circulaire vient du logo préexistant, qui suit la forme du C et d’une mappemonde.\\
Les couleurs de ce site sont très vives, ce qui peut constituer une surcharge visuelle, c’est pourquoi une image plus douce sert d’arrière\sphinxhyphen{}plan en haut de chaque page du site Web. Cette image correspond parfaitement à l’identité visuelle de Candide et du site en règle générale : prédominance de vert plutôt neutre, touches d’orange et présence de fleurs qui rappelle l’importance de la protection de la nature pour le comité.

\begin{figure}[htbp]
\centering
\capstart

\noindent\sphinxincludegraphics[width=1.000\linewidth]{{moodboard}.jpg}
\caption{MoodBoard du comité étudiant humanitaire du Collège du Sud}\label{\detokenize{chapitre-06:id1}}\end{figure}


\subsection{Maquettes}
\label{\detokenize{chapitre-06:maquettes}}
\sphinxAtStartPar
Trois prototypes du site Web de Candide sont réalisés dans le cadre de ce Travail de Maturité et sont sujets à de nombreuses discussions afin d’arriver à un choix final. Chacune a connu de multiples modifications dans le but d’être rendue fidèle à l’image du comité et à celle que le comité décide de véhiculer.


\subsubsection{Première maquette}
\label{\detokenize{chapitre-06:premiere-maquette}}
\sphinxAtStartPar
À partir des informations dont Candide a fait part, démarre la conception d’une première maquette. Celle\sphinxhyphen{}ci est directement basée sur le moodboard et cherche à être au plus proche de leurs souhaits.

\sphinxAtStartPar
Suite à cela, deux autres maquettes plus libres sont inventées, afin d’explorer des nouvelles pistes.


\subsubsection{Première maquette « d’exploration »}
\label{\detokenize{chapitre-06:premiere-maquette-dexploration}}
\sphinxAtStartPar
La première maquette « d’exploration » est plutôt sobre dans ses couleurs. Cependant, les polices apportent l’aspect plus décontracté du comité étudiant. La structure du site, quant à elle, reste la même et seuls certains éléments de la mise\sphinxhyphen{}en\sphinxhyphen{}page ont été modifiés. Avec ce nouveau prototype, la page de présentation des comités au fil des années est beaucoup plus claire.\\
Malheureusement, les couleurs restent trop ternes et les membres de Candide aimeraient donner à leur site un aspect moins grisâtre, pour mettre en avant la bienveillance du comité de façon plus marquée.

\begin{figure}[htbp]
\centering
\capstart

\noindent\sphinxincludegraphics[width=1.000\linewidth]{{Proto2}.png}
\caption{Première maquette « d’exploration »}\label{\detokenize{chapitre-06:id2}}\end{figure}


\subsubsection{Deuxième maquette « d’exploration »}
\label{\detokenize{chapitre-06:deuxieme-maquette-dexploration}}
\sphinxAtStartPar
La seconde maquette « d’exploration » amène un aspect plus moderne et classique que la précédente. Un vert sombre domine et met en avant le souhait de Candide d’agir pour la préservation de l’environnement. Le côté atténué de la couleur donne, quant à lui, une image beaucoup plus professionnelle au comité.  En revanche, après discussion avec les étudiants, ils ne désirent pas mettre cela en avant. Il est effectivement important de garder en tête qu’il s’agit d’un comité étudiant et ne pas rendre ce site trop classique.

\begin{figure}[htbp]
\centering
\capstart

\noindent\sphinxincludegraphics[width=1.000\linewidth]{{Proto3}.png}
\caption{Deuxième maquette « d’exploration »}\label{\detokenize{chapitre-06:id3}}\end{figure}


\subsubsection{Choix final}
\label{\detokenize{chapitre-06:choix-final}}
\sphinxAtStartPar
Après plusieurs discussions et modifications de chaque maquette, l’idée sélectionnée est la première. Elle est celle qui correspond d’ailleurs le plus à la description de ce que Candide veut, bien qu’elle ait subi beaucoup de modifications pour la rendre plus sobre, mais tout en restant moderne. Les changements apportés sont d’ailleurs largement inspirés du troisième prototype dans le but d’apporter une dimension beaucoup plus professionnelle au site.

\begin{figure}[htbp]
\centering
\capstart

\noindent\sphinxincludegraphics[width=1.000\linewidth]{{Proto_final}.png}
\caption{Maquette finale du site de Candide}\label{\detokenize{chapitre-06:id4}}\end{figure}


\chapter{Tutoriel Sphinx / MyST}
\label{\detokenize{tutoriel-sphinx/index:tutoriel-sphinx-myst}}\label{\detokenize{tutoriel-sphinx/index::doc}}
\begin{sphinxadmonition}{note}{En bref}

\sphinxAtStartPar
Sphinx est le système de documentation du langage Python (\sphinxurl{https://www.sphinx-doc.org/en/master/}).

\sphinxAtStartPar
Ce tutoriel, livré avec le kit de démarrage Sphinx pour la rédaction du travail écrit de votre TM, vous permet d’acquérir rapidement les bases de la syntaxe Sphinx.
\end{sphinxadmonition}


\section{Fonctionnement de la toolchain Sphinx}
\label{\detokenize{tutoriel-sphinx/fonctionnement:fonctionnement-de-la-toolchain-sphinx}}\label{\detokenize{tutoriel-sphinx/fonctionnement::doc}}
\sphinxAtStartPar
Avant de commencer, il faut comprendre le fonctionnement général de Sphinx. Il
s’agit d’une « toolchain », à savoir un ensemble d’outils qui permettent de
transformer des fichiers Markdown en documents formatés (soit HTML pour le Web
ou LaTeX pour produire un PDF).

\sphinxAtStartPar
Pour les personnes qui connaissent un peu le LaTeX, Sphinx est un système un peu
analogue: on écrit du code et un outil s’occupe de tout mettre en page et faire
la reliure de manière professionnelle. D’ailleurs, Sphinx permet de générer du
LaTeX pour générer un document PDF.

\sphinxAtStartPar
Au contraire de Microsoft Word, où l’on voit directement le résultat final
lorsqu’on écrit le contenu, Sphinx utilise un paradigme semblable à celui
utilisé en LaTeX: on écrit du code avec une syntaxe particulière qui est ensuite
transformé dans le résultat final par la toolchain.

\begin{figure}[htbp]
\centering
\capstart

\noindent\sphinxincludegraphics[width=1.000\linewidth]{{tutoriel-sphinx/figures/make-html}.png}
\caption{Processus de génération du HTML avec Sphinx}\label{\detokenize{tutoriel-sphinx/fonctionnement:id1}}\end{figure}

\begin{figure}[htbp]
\centering
\capstart

\noindent\sphinxincludegraphics[width=1.000\linewidth]{{tutoriel-sphinx/figures/make-tmpdf}.png}
\caption{Processus de génération du PDF avec Sphinx}\label{\detokenize{tutoriel-sphinx/fonctionnement:id2}}\end{figure}


\subsection{Les avantages de Sphinx par rapport à Word}
\label{\detokenize{tutoriel-sphinx/fonctionnement:les-avantages-de-sphinx-par-rapport-a-word}}
\sphinxAtStartPar
Par rapport à Word, Sphinx présente les avantages suivants
\begin{itemize}
\item {} 
\sphinxAtStartPar
Sphinx est un outil Open\sphinxhyphen{}Source, ce qui signifie qu’on peut l’adapter sans
problème à n’importe quel besoin. De plus, il n’y a aucun risque que le
système ne soit tout\sphinxhyphen{}à\sphinxhyphen{}coup plus supporté, car une importante communauté s’occupe de le faire évoluer et de corriger les bugs détectés.

\item {} 
\sphinxAtStartPar
On écrit de la documentation dans des fichiers qui ne contiennent que du
texte, comme une sorte de « code source ».

\item {} 
\sphinxAtStartPar
On écrit le contenu dans Visual Studio Code ou un autre éditeur. On peut donc
utiliser toute la puissance de tels éditeurs pour optimiser la rédaction du contenu.

\item {} 
\sphinxAtStartPar
Le contenu est très facilement versionalisable (utilisation de git et Github).

\item {} 
\sphinxAtStartPar
Permet d’intégrer très facilement les éléments suivants qui sont très courants
dans un travail de maturité en informatique
\begin{itemize}
\item {} 
\sphinxAtStartPar
Des mathématiques

\item {} 
\sphinxAtStartPar
Du code source

\item {} 
\sphinxAtStartPar
Des éléments interactifs (quiz, programmes qui s’exécutent directement dans
le navigateur, etc. )

\end{itemize}

\item {} 
\sphinxAtStartPar
À partir d’une seule source, on peut générer des documents pour différentes
cibles (Site Web, Document PDF, e\sphinxhyphen{}book pour une liseuse, …)

\item {} 
\sphinxAtStartPar
Le système est extensible et facilement personnalisable

\item {} 
\sphinxAtStartPar
Lorsqu’on écrit un document pour Sphinx, il n’est pas nécessaire de se
préoccuper de la mise en page et de la mise en forme du contenu. On se
contente d’écrire

\end{itemize}


\subsection{Les désavantages de Sphinx par rapport à Word}
\label{\detokenize{tutoriel-sphinx/fonctionnement:les-desavantages-de-sphinx-par-rapport-a-word}}\begin{itemize}
\item {} 
\sphinxAtStartPar
L’utilisation de Sphinx est moins intuitive que celle de Word au début.

\item {} 
\sphinxAtStartPar
Word est un traitement de texte de type « WYSIWYG » (What You See Is What You
Get). Lorsqu’on écrit le contenu dans Word, on voit directement le résultat
final. Sphinx est un système de rédaction de type « WYSIWYM » (What You See Is
What You Mean). Comme on ne voit pas directement le résultat final, il faut
être attentif lors de la relecture du résultat final. On peut parfois avoir de sacrées surprises, en raison d’une erreur de syntaxe par exemple.

\item {} 
\sphinxAtStartPar
La correction orthographique est moins facile qu’en Word, car il n’existe pas
d’outil parfaitement intégré qui permette de faire la correction de manière
automatique. Il est toutefois possible de passer du Sphinx à Word (avec un rendu dégradé), pour utiliser le correcteur automatique de Microsoft Word.

\end{itemize}


\section{Les bases de la rédaction avec Sphinx}
\label{\detokenize{tutoriel-sphinx/bases:les-bases-de-la-redaction-avec-sphinx}}\label{\detokenize{tutoriel-sphinx/bases:sphinx-bases}}\label{\detokenize{tutoriel-sphinx/bases::doc}}
\begin{sphinxadmonition}{note}{En cours de rédaction}

\sphinxAtStartPar
Cette section est en cours de rédaction et sera complétée à l’avenir.
\end{sphinxadmonition}


\subsection{Fichiers Markdown ou RestructuredText}
\label{\detokenize{tutoriel-sphinx/bases:fichiers-markdown-ou-restructuredtext}}
\sphinxAtStartPar
Le contenu à promprement parler est écrit dans des fichiers textes au format
Markdown / MyST (voir)

\begin{sphinxadmonition}{tip}{Astuce:}
\sphinxAtStartPar
Pour écrire le TM, il est conseillé d’utiliser le format MyST qui

\sphinxAtStartPar
dispose d’une \sphinxhref{https://marketplace.visualstudio.com/items?itemName=ExecutableBookProject.myst-highlight}{extension Visual Studio Code très pratique} (https://marketplace.visualstudio.com/items?itemName=ExecutableBookProject.myst\sphinxhyphen{}highlight).
\end{sphinxadmonition}


\subsection{Bien démarrer}
\label{\detokenize{tutoriel-sphinx/bases:bien-demarrer}}
\sphinxAtStartPar
Pour le moment, il est recommandé de travailler sur Gitpod pour rédiger votre
TM, à moins que vous ayez Linux installé sur votre machine et que vous soyez à
l’aise avec.

\begin{sphinxadmonition}{note}{Information}

\sphinxAtStartPar
Nous allons apprendre à utiliser Linux ces prochains mois au cours d’OC.
\end{sphinxadmonition}

\sphinxAtStartPar
En attendant que ce tutoriel soit plus fourni, reportez\sphinxhyphen{}vous à la documentation
officielle de MyST
\begin{itemize}
\item {} 
\sphinxAtStartPar
https://myst\sphinxhyphen{}parser.readthedocs.io/en/latest/sphinx/intro.html\#intro\sphinxhyphen{}writing

\item {} 
\sphinxAtStartPar
https://myst\sphinxhyphen{}parser.readthedocs.io/en/latest/syntax/syntax.html\#

\item {} 
\sphinxAtStartPar
https://myst\sphinxhyphen{}parser.readthedocs.io/en/latest/syntax/optional.html

\item {} 
\sphinxAtStartPar
https://myst\sphinxhyphen{}parser.readthedocs.io/en/latest/syntax/reference.html

\end{itemize}

\sphinxAtStartPar
Il peut également être utile de consulter la documentation de Sphinx pour savoir
comment utiliser les fonctionnalités avancées.
\begin{itemize}
\item {} 
\sphinxAtStartPar
https://www.sphinx\sphinxhyphen{}doc.org/en/master/index.html

\end{itemize}


\section{Insérer du code}
\label{\detokenize{tutoriel-sphinx/inserer-code:inserer-du-code}}\label{\detokenize{tutoriel-sphinx/inserer-code::doc}}
\begin{sphinxadmonition}{note}{Objectifs de cette page}

\sphinxAtStartPar
Cette page montre comment insérer du code Python, HTML ou Javascript dans la documentation.
\end{sphinxadmonition}

\begin{sphinxadmonition}{note}{Remarque}

\sphinxAtStartPar
Ma nouvelle remarque
\end{sphinxadmonition}


\subsection{Insertion simple}
\label{\detokenize{tutoriel-sphinx/inserer-code:insertion-simple}}
\sphinxAtStartPar
Le plus simple pour insérer du code dans la documentation est d’utiliser les outils natifs mis à disposition par Markdown pour cela. Cela ne laisse cependant que peu de marge de manoeuvre de personnalisation.

\begin{sphinxVerbatim}[commandchars=\\\{\}]
\PYG{k+kn}{from} \PYG{n+nn}{math} \PYG{k+kn}{import} \PYG{n}{pi}

\PYG{k}{def} \PYG{n+nf}{foo}\PYG{p}{(}\PYG{n}{a}\PYG{p}{,} \PYG{n}{b}\PYG{p}{)}\PYG{p}{:}
    \PYG{k}{return} \PYG{n}{a} \PYG{o}{+} \PYG{n}{b}

\PYG{n+nb}{print}\PYG{p}{(}\PYG{n}{foo}\PYG{p}{(}\PYG{l+m+mi}{10}\PYG{p}{,} \PYG{l+m+mi}{20}\PYG{p}{)}\PYG{p}{)}
\end{sphinxVerbatim}

\begin{sphinxVerbatim}[commandchars=\\\{\}]
\PYG{k+kd}{let} \PYG{n+nx}{maVariable} \PYG{o}{=} \PYG{l+m+mf}{3}\PYG{p}{;}
\end{sphinxVerbatim}

\begin{sphinxVerbatim}[commandchars=\\\{\}]
\PYG{p}{\PYGZlt{}}\PYG{n+nt}{html}\PYG{p}{\PYGZgt{}}
    \PYG{p}{\PYGZlt{}}\PYG{n+nt}{body}\PYG{p}{\PYGZgt{}}
    \PYG{p}{\PYGZlt{}}\PYG{p}{/}\PYG{n+nt}{body}\PYG{p}{\PYGZgt{}}
\PYG{p}{\PYGZlt{}}\PYG{p}{/}\PYG{n+nt}{html}\PYG{p}{\PYGZgt{}}
\end{sphinxVerbatim}

\begin{sphinxadmonition}{note}{Code Markdown}

\begin{sphinxVerbatim}[commandchars=\\\{\}]
\PYG{l+s+sb}{```}\PYG{l+s+sb}{python}
\PYG{k+kn}{from} \PYG{n+nn}{math} \PYG{k+kn}{import} \PYG{n}{pi}

\PYG{k}{def} \PYG{n+nf}{foo}\PYG{p}{(}\PYG{n}{a}\PYG{p}{,} \PYG{n}{b}\PYG{p}{)}\PYG{p}{:}
    \PYG{k}{return} \PYG{n}{a} \PYG{o}{+} \PYG{n}{b}

\PYG{n+nb}{print}\PYG{p}{(}\PYG{n}{foo}\PYG{p}{(}\PYG{l+m+mi}{10}\PYG{p}{,} \PYG{l+m+mi}{20}\PYG{p}{)}\PYG{p}{)}
\PYG{l+s+sb}{```}
\end{sphinxVerbatim}
\end{sphinxadmonition}


\subsection{Plus de flexibilité}
\label{\detokenize{tutoriel-sphinx/inserer-code:plus-de-flexibilite}}
\sphinxAtStartPar
On peut également utiliser la direction Sphinx \sphinxcode{\sphinxupquote{code\sphinxhyphen{}block}} pour insérer du code de manière plus sophistiquée, avec de la numérotation automatique, des lignes mises en évidence et des tas d’autres options (voir)

\fvset{hllines={, 3, 4,}}%
\begin{sphinxVerbatim}[commandchars=\\\{\},numbers=left,firstnumber=1,stepnumber=1]
\PYG{k+kn}{from} \PYG{n+nn}{math} \PYG{k+kn}{import} \PYG{n}{pi}

\PYG{k}{def} \PYG{n+nf}{foo}\PYG{p}{(}\PYG{n}{a}\PYG{p}{,} \PYG{n}{b}\PYG{p}{)}\PYG{p}{:}
    \PYG{k}{return} \PYG{n}{a} \PYG{o}{+} \PYG{n}{b}
\end{sphinxVerbatim}
\sphinxresetverbatimhllines

\begin{sphinxadmonition}{note}{Code Markdown}

\begin{sphinxVerbatim}[commandchars=\\\{\}]
\PYG{g+gu}{```\PYGZob{}code\PYGZhy{}block\PYGZcb{} python}
\PYG{g+gu}{\PYGZhy{}\PYGZhy{}\PYGZhy{}}
emphasize\PYGZhy{}lines: 1
\PYG{g+gu}{linenos: true}
\PYG{g+gu}{\PYGZhy{}\PYGZhy{}\PYGZhy{}}
from math import pi

def foo(a, b):
    return a + b
```
\end{sphinxVerbatim}
\end{sphinxadmonition}


\subsection{Insérer du code depuis un fichier externe}
\label{\detokenize{tutoriel-sphinx/inserer-code:inserer-du-code-depuis-un-fichier-externe}}
\sphinxAtStartPar
Il est également possible d’inclure du code dans la documentation depuis un fichier externe, au lieu d’avoir à écrire copier le code directement dans le fichier \sphinxcode{\sphinxupquote{.md}}.

\fvset{hllines={, 1,}}%
\begin{sphinxVerbatim}[commandchars=\\\{\},numbers=left,firstnumber=1,stepnumber=1]
\PYG{k+kn}{from} \PYG{n+nn}{gturtle} \PYG{k+kn}{import} \PYG{o}{*}

\PYG{n}{makeTurtle}\PYG{p}{(}\PYG{p}{)}

\PYG{n}{forward}\PYG{p}{(}\PYG{l+m+mi}{100}\PYG{p}{)}
\end{sphinxVerbatim}
\sphinxresetverbatimhllines


\subsubsection{Titre 2}
\label{\detokenize{tutoriel-sphinx/inserer-code:titre-2}}

\paragraph{Titre 3}
\label{\detokenize{tutoriel-sphinx/inserer-code:titre-3}}

\section{Insérer des images}
\label{\detokenize{tutoriel-sphinx/inserer-figures:inserer-des-images}}\label{\detokenize{tutoriel-sphinx/inserer-figures:chapter-insert-images}}\label{\detokenize{tutoriel-sphinx/inserer-figures::doc}}
\begin{figure}[htbp]
\centering
\capstart

\noindent\sphinxincludegraphics[width=1.000\linewidth]{{tutoriel-sphinx/figures/make-html}.png}
\caption{Une légende}\label{\detokenize{tutoriel-sphinx/inserer-figures:id1}}\end{figure}


\subsection{Insérer un tableau}
\label{\detokenize{tutoriel-sphinx/inserer-figures:inserer-un-tableau}}

\section{Fonctionnalités avancées}
\label{\detokenize{tutoriel-sphinx/advanced-features:fonctionnalites-avancees}}\label{\detokenize{tutoriel-sphinx/advanced-features:advanced-features}}\label{\detokenize{tutoriel-sphinx/advanced-features::doc}}

\subsection{Encadrés}
\label{\detokenize{tutoriel-sphinx/advanced-features:encadres}}
\begin{sphinxShadowBox}
\sphinxstylesidebartitle{Encadré mis en évidence}

\sphinxAtStartPar
Ceci est un encadré qui sort du fil du texte (on dit que cet encadré et
« flottant », car il flotte sur le texte principal).

\sphinxAtStartPar
La directive Sphinx \sphinxcode{\sphinxupquote{sidebar}} permet d’insérer certaines informations dans un
encadré (dans le PDF) et dans une sorte de panneau flottant dans la version
HTML.
\begin{equation*}
\begin{split}f(x) = 2x^2 + 1\end{split}
\end{equation*}
\sphinxAtStartPar
Il faut commencer par placer le \sphinxcode{\sphinxupquote{sidebar}} dans le code Markdown et ensuite
placer le contenu qui viendra se placer à gauche de la page.
\end{sphinxShadowBox}

\sphinxAtStartPar
Paragraphe est écrit de manière normale, au fil du texte.
\begin{itemize}
\item {} 
\sphinxAtStartPar
On pourrait remplir avec du contenu …. mais ce contenu n’est pas très
intéressant. Ce contenu peut faire plusieurs lignes dans le fichier Markdown.
C’est tout de même un peu long.

\end{itemize}

\begin{sphinxadmonition}{warning}{Avertissement:}
\sphinxAtStartPar
Il ne faut pas trop abuser des « sidebars », car le placement n’est pas du tout le même sur la version HTML en ligne et dans la version PDF.

\sphinxAtStartPar
Il vaut éviter également de mettre trop de contenu dans les \sphinxcode{\sphinxupquote{sidebar}}.
\end{sphinxadmonition}


\subsection{Direction \sphinxstyleliteralintitle{\sphinxupquote{raw}}}
\label{\detokenize{tutoriel-sphinx/advanced-features:direction-raw}}
\begin{sphinxadmonition}{tip}{Astuce:}
\sphinxAtStartPar
La directive \sphinxcode{\sphinxupquote{raw}} permet de différencier le contenu présenté dans la version HTML et celui présent dans la version PDF. Elle permet également d’insérer du contenu tel quel pour profiter de toutes les fonctionnalités du format cible (par exemple le HTML ou le LaTeX).
\end{sphinxadmonition}

\sphinxAtStartPar
Parfois, il est utile d’inclure un certain contenu (images, vidéos, …) dans la
version HTML en ligne, mais de ne pas inclure ce contenu dans la version PDF.
C’est typiquement le cas pour des vidéos ou d’autres éléments interactifs.



En revanche, ce contenu n'apparaîtra que dans le LaTeX et pas dans le HTML.


\begin{sphinxadmonition}{tip}{Astuce:}
\sphinxAtStartPar
L’avantage de la directive \sphinxcode{\sphinxupquote{raw}} est de pouvoir insérer n’importe quel code
HTML, aussi exotique soit\sphinxhyphen{}il.
\end{sphinxadmonition}


\section{Références croisées}
\label{\detokenize{tutoriel-sphinx/cross-references:references-croisees}}\label{\detokenize{tutoriel-sphinx/cross-references::doc}}
\sphinxAtStartPar
On peut faire référence à un autre chapitre \DUrole{xref,std,std-ref}{chapter\sphinxhyphen{}insert\sphinxhyphen{}images}.


\chapter{Indices and tables}
\label{\detokenize{index:indices-and-tables}}\begin{itemize}
\item {} 
\sphinxAtStartPar
\DUrole{xref,std,std-ref}{genindex}

\item {} 
\sphinxAtStartPar
\DUrole{xref,std,std-ref}{search}

\end{itemize}



\renewcommand{\indexname}{Index}
\printindex
\end{document}